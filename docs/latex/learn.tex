To learn how to use {\bfseries {\itshape vt}}, there are several resources that will help.


\begin{DoxyItemize}
\item Learn \hyperlink{vt-build}{How to Build}
\item Walk through the \hyperlink{tutorial}{Tutorial}
\item Read the small example programs
\begin{DoxyItemize}
\item Read the \hyperlink{examples}{Examples} walk-\/through
\item Examine {\ttfamily vt/examples/$\ast$}
\end{DoxyItemize}
\item Read about the different \hyperlink{introduction_vt-components}{Components in vt}
\item Messaging relies on serializing data with the {\itshape checkpoint} library. Learn \hyperlink{ckpt_learn_serialize}{How to Serialize Data} 
\end{DoxyItemize}\hypertarget{vt-build}{}\section{How to Build}\label{vt-build}
Building D\+A\+R\+M\+A/vt with cmake

{\bfseries {\itshape vt}} can be built with {\ttfamily cmake} or built inside a {\ttfamily docker} container. Most of the external dependencies come bundled with {\bfseries {\itshape vt}} for ease of compiling.\hypertarget{vt-build_how-to-build}{}\subsection{Building}\label{vt-build_how-to-build}
To build {\bfseries {\itshape vt}}, one must obtain the following dependencies\+:\hypertarget{vt-build_required-deps}{}\subsubsection{Required}\label{vt-build_required-deps}

\begin{DoxyItemize}
\item checkpoint, ({\itshape vt} ecosystem)
\item M\+PI (mpich/openmpi/mvapich/\+I\+BM Spectrum M\+P\+I/\+Cray M\+P\+I\+C\+H/etc.)
\end{DoxyItemize}\hypertarget{vt-build_automatic-build-deps}{}\subsubsection{Automatically build dependencies}\label{vt-build_automatic-build-deps}
Assuming M\+PI is installed and accessible via C\+C/\+C\+XX, the only other dependency that is required is checkpoint. The easiest way to get these built are to clone it inside {\ttfamily vt/lib}\+:


\begin{DoxyCode}
$ git clone git@github.com:DARMA-tasking/vt
$ cd vt/lib
$ git clone git@github.com:DARMA-tasking/checkpoint
\end{DoxyCode}


With these in {\ttfamily vt/lib}, cmake will automatically build them and stitch them into {\itshape vt}\textquotesingle{}s linking process.\hypertarget{vt-build_use-cmake-directly-vars}{}\subsubsection{Using cmake directly}\label{vt-build_use-cmake-directly-vars}
One may use {\ttfamily cmake} as normal on {\itshape vt}, with checkpoint cloned in {\ttfamily vt/lib} to compile them all together as explained above. The following are some custom configuration build options that can be provided to {\ttfamily cmake} to change the build configuration\+:

\tabulinesep=1mm
\begin{longtabu} spread 0pt [c]{*{3}{|X[-1]}|}
\hline
\rowcolor{\tableheadbgcolor}\textbf{ C\+Make Variable }&\textbf{ Default Value }&\textbf{ Description  }\\\cline{1-3}
\endfirsthead
\hline
\endfoot
\hline
\rowcolor{\tableheadbgcolor}\textbf{ C\+Make Variable }&\textbf{ Default Value }&\textbf{ Description  }\\\cline{1-3}
\endhead
{\ttfamily vt\+\_\+lb\+\_\+enabled} &1 &Compile with support for runtime load balancing \\\cline{1-3}
{\ttfamily vt\+\_\+trace\+\_\+enabled} &0 &Compile with support for runtime tracing (Projections-\/format) \\\cline{1-3}
{\ttfamily vt\+\_\+trace\+\_\+only} &0 &Compile vt in trace-\/only mode (stripped down version for tracing M\+PI calls) \\\cline{1-3}
{\ttfamily vt\+\_\+test\+\_\+trace\+\_\+runtime\+\_\+enabled} &0 &Force tracing on at runtime for VT tests \\\cline{1-3}
{\ttfamily vt\+\_\+doxygen\+\_\+enabled} &0 &Enable doxygen generation \\\cline{1-3}
{\ttfamily vt\+\_\+mimalloc\+\_\+enabled} &0 &Enable {\ttfamily mimalloc}, alternative allocator for debugging memory usage/frees/corruption \\\cline{1-3}
{\ttfamily vt\+\_\+asan\+\_\+enabled} &0 &Enable building with address sanitizer \\\cline{1-3}
{\ttfamily vt\+\_\+ubsan\+\_\+enabled} &0 &Enable building with undefined behavior sanitizer \\\cline{1-3}
{\ttfamily vt\+\_\+werror\+\_\+enabled} &0 &Treat all warnings as errors \\\cline{1-3}
{\ttfamily vt\+\_\+pool\+\_\+enabled} &1 &Use memory pool in {\itshape vt} for message allocation \\\cline{1-3}
{\ttfamily vt\+\_\+zoltan\+\_\+enabled} &0 &Build with Zoltan enabled for {\ttfamily Zoltan\+LB} support \\\cline{1-3}
{\ttfamily vt\+\_\+mpi\+\_\+guards} &0 &Guards against mis-\/use of M\+PI calls in code using {\itshape vt} \\\cline{1-3}
{\ttfamily vt\+\_\+priorities\+\_\+enabled} &1 &Enable prioritization of work (adds bits in envelope) \\\cline{1-3}
{\ttfamily vt\+\_\+diagnostics\+\_\+enabled} &1 &Enable VT component diagnostics for performance analysis \\\cline{1-3}
{\ttfamily vt\+\_\+diagnostics\+\_\+runtime\+\_\+enabled} &0 &Enable VT component diagnostics at runtime by default \\\cline{1-3}
{\ttfamily vt\+\_\+priority\+\_\+bits\+\_\+per\+\_\+level} &3 &Number of bits per level of priority in envelope \\\cline{1-3}
{\ttfamily vt\+\_\+build\+\_\+extended\+\_\+tests} &1 &Build with full, extended testing \\\cline{1-3}
{\ttfamily vt\+\_\+production\+\_\+build\+\_\+enabled} &0 &Disable assertions and debug prints at compile time \\\cline{1-3}
{\ttfamily vt\+\_\+unity\+\_\+build\+\_\+enabled} &0 &Build with Unity/\+Jumbo mode enabled (requires C\+Make $>$= 3.\+16) \\\cline{1-3}
{\ttfamily vt\+\_\+fcontext\+\_\+enabled} &0 &Force use of fcontext for threading \\\cline{1-3}
{\ttfamily vt\+\_\+tests\+\_\+num\+\_\+nodes} &-\/ &Maximum number of nodes used for tests. If empty, then the default value detected by C\+Make is used \\\cline{1-3}
{\ttfamily C\+O\+D\+E\+\_\+\+C\+O\+V\+E\+R\+A\+GE} &0 &Enable code coverage for VT examples/tests \\\cline{1-3}
{\ttfamily V\+T\+\_\+\+B\+U\+I\+L\+D\+\_\+\+T\+E\+S\+TS} &1 &Build all VT tests \\\cline{1-3}
{\ttfamily V\+T\+\_\+\+B\+U\+I\+L\+D\+\_\+\+E\+X\+A\+M\+P\+L\+ES} &1 &Build all VT examples \\\cline{1-3}
{\ttfamily vt\+\_\+debug\+\_\+verbose} &1 (not Release) &Enable VT verbose debug prints at compile-\/time \\\cline{1-3}
{\ttfamily vt\+\_\+no\+\_\+color\+\_\+enabled} &0 &Set {\ttfamily -\/-\/vt\+\_\+no\+\_\+color} flag to true by default \\\cline{1-3}
{\ttfamily B\+U\+I\+L\+D\+\_\+\+S\+H\+A\+R\+E\+D\+\_\+\+L\+I\+BS} &0 &Build VT as shared library \\\cline{1-3}
\end{longtabu}
\hypertarget{vt-build_using-the-build-script}{}\subsubsection{Using the Build Script}\label{vt-build_using-the-build-script}
Instead of running {\ttfamily cmake}, one may invoke the {\ttfamily vt/ci/build\+\_\+cpp.\+sh} script which will run {\ttfamily cmake} for {\itshape vt} with environment variables for most configuration parameters.\hypertarget{vt-build_building-environment-variables}{}\paragraph{Build Script Environment Variables}\label{vt-build_building-environment-variables}
\tabulinesep=1mm
\begin{longtabu} spread 0pt [c]{*{3}{|X[-1]}|}
\hline
\rowcolor{\tableheadbgcolor}\textbf{ Variable }&\textbf{ Default Value }&\textbf{ Description  }\\\cline{1-3}
\endfirsthead
\hline
\endfoot
\hline
\rowcolor{\tableheadbgcolor}\textbf{ Variable }&\textbf{ Default Value }&\textbf{ Description  }\\\cline{1-3}
\endhead
{\ttfamily C\+M\+A\+K\+E\+\_\+\+B\+U\+I\+L\+D\+\_\+\+T\+Y\+PE} &Release &The {\ttfamily cmake} build type \\\cline{1-3}
{\ttfamily V\+T\+\_\+\+L\+B\+\_\+\+E\+N\+A\+B\+L\+ED} &1 &Compile with support for runtime load balancing \\\cline{1-3}
{\ttfamily V\+T\+\_\+\+T\+R\+A\+C\+E\+\_\+\+E\+N\+A\+B\+L\+ED} &0 &Compile with support for runtime tracing (Projections-\/format) \\\cline{1-3}
{\ttfamily V\+T\+\_\+\+T\+R\+A\+C\+E\+\_\+\+O\+N\+LY} &0 &Compile vt in trace-\/only mode (stripped down version for tracing M\+PI calls) \\\cline{1-3}
{\ttfamily V\+T\+\_\+\+T\+R\+A\+C\+E\+\_\+\+R\+U\+N\+T\+I\+M\+E\+\_\+\+E\+N\+A\+B\+L\+ED} &0 &Force tracing on at runtime (used in CI for automatically testing tracing on all tests/examples) \\\cline{1-3}
{\ttfamily V\+T\+\_\+\+D\+O\+X\+Y\+G\+E\+N\+\_\+\+E\+N\+A\+B\+L\+ED} &0 &Enable doxygen generation \\\cline{1-3}
{\ttfamily V\+T\+\_\+\+M\+I\+M\+A\+L\+L\+O\+C\+\_\+\+E\+N\+A\+B\+L\+ED} &0 &Enable {\ttfamily mimalloc}, alternative allocator for debugging memory usage/frees/corruption \\\cline{1-3}
{\ttfamily V\+T\+\_\+\+A\+S\+A\+N\+\_\+\+E\+N\+A\+B\+L\+ED} &0 &Enable building with address sanitizer \\\cline{1-3}
{\ttfamily V\+T\+\_\+\+U\+B\+S\+A\+N\+\_\+\+E\+N\+A\+B\+L\+ED} &0 &Enable building with undefined behavior sanitizer \\\cline{1-3}
{\ttfamily V\+T\+\_\+\+W\+E\+R\+R\+O\+R\+\_\+\+E\+N\+A\+B\+L\+ED} &0 &Treat all warnings as errors \\\cline{1-3}
{\ttfamily V\+T\+\_\+\+P\+O\+O\+L\+\_\+\+E\+N\+A\+B\+L\+ED} &1 &Use memory pool in {\itshape vt} for message allocation \\\cline{1-3}
{\ttfamily V\+T\+\_\+\+F\+C\+O\+N\+T\+E\+X\+T\+\_\+\+E\+N\+A\+B\+L\+ED} &0 &Force use of fcontext for threading \\\cline{1-3}
{\ttfamily V\+T\+\_\+\+Z\+O\+L\+T\+A\+N\+\_\+\+E\+N\+A\+B\+L\+ED} &0 &Build with Zoltan enabled for {\ttfamily Zoltan\+LB} support \\\cline{1-3}
{\ttfamily Z\+O\+L\+T\+A\+N\+\_\+\+D\+IR} &$<$empty$>$ &Directory pointing to Zoltan installation \\\cline{1-3}
{\ttfamily V\+T\+\_\+\+M\+P\+I\+\_\+\+G\+U\+A\+R\+D\+\_\+\+E\+N\+A\+B\+L\+ED} &0 &Guards against mis-\/use of M\+PI calls in code using {\itshape vt} \\\cline{1-3}
{\ttfamily V\+T\+\_\+\+E\+X\+T\+E\+N\+D\+E\+D\+\_\+\+T\+E\+S\+T\+S\+\_\+\+E\+N\+A\+B\+L\+ED} &1 &Build with full, extended testing \\\cline{1-3}
{\ttfamily V\+T\+\_\+\+U\+N\+I\+T\+Y\+\_\+\+B\+U\+I\+L\+D\+\_\+\+E\+N\+A\+B\+L\+ED} &0 &Build with Unity/\+Jumbo mode enabled (requires C\+Make $>$= 3.\+16) \\\cline{1-3}
{\ttfamily V\+T\+\_\+\+P\+R\+O\+D\+U\+C\+T\+I\+O\+N\+\_\+\+B\+U\+I\+L\+D\+\_\+\+E\+N\+A\+B\+L\+ED} &0 &Disable assertions and debug prints at compile time \\\cline{1-3}
{\ttfamily V\+T\+\_\+\+D\+I\+A\+G\+N\+O\+S\+T\+I\+C\+S\+\_\+\+E\+N\+A\+B\+L\+ED} &1 &Enable VT component diagnostics for performance analysis \\\cline{1-3}
{\ttfamily V\+T\+\_\+\+D\+I\+A\+G\+N\+O\+S\+T\+I\+C\+S\+\_\+\+R\+U\+N\+T\+I\+M\+E\+\_\+\+E\+N\+A\+B\+L\+ED} &0 &Enable VT component diagnostics at runtime by default \\\cline{1-3}
{\ttfamily V\+T\+\_\+\+D\+E\+B\+U\+G\+\_\+\+V\+E\+R\+B\+O\+SE} &$<$empty$>$ &Enable VT verbose debug prints at compile-\/time \\\cline{1-3}
{\ttfamily V\+T\+\_\+\+T\+E\+S\+T\+S\+\_\+\+N\+U\+M\+\_\+\+N\+O\+D\+ES} &$<$empty$>$ &Maximum number of nodes used for tests. If empty, then the default value detected by C\+Make is used \\\cline{1-3}
{\ttfamily V\+T\+\_\+\+N\+O\+\_\+\+C\+O\+L\+O\+R\+\_\+\+E\+N\+A\+B\+L\+ED} &0 &Set {\ttfamily -\/-\/vt\+\_\+no\+\_\+color} flag to true by default \\\cline{1-3}
{\ttfamily B\+U\+I\+L\+D\+\_\+\+S\+H\+A\+R\+E\+D\+\_\+\+L\+I\+BS} &0 &Build VT as shared library \\\cline{1-3}
\end{longtabu}
With these set, invoke the script with two arguments\+: the path to the {\itshape vt} root directory and the build path. Here\textquotesingle{}s an example assuming that {\itshape vt} is cloned into {\ttfamily /usr/src/vt} with trace enabled in debug mode.

{\bfseries Usage for building\+:}


\begin{DoxyCode}
$ vt/ci/build\_cpp.sh <full-path-to-vt-source> <full-path-to-build-dir>
\end{DoxyCode}


{\bfseries Example\+:}


\begin{DoxyCode}
$ cd /usr/src
$ git clone git@github.com:DARMA-tasking/vt
$ VT\_TRACE\_ENABLED=1 CMAKE\_BUILD\_TYPE=Debug /usr/src/vt/ci/build\_cpp.sh /usr/src/vt /usr/build/vt
\end{DoxyCode}
\hypertarget{vt-build_docker-build}{}\subsubsection{Building with `docker` containerization}\label{vt-build_docker-build}
The easiest way to build {\itshape vt} is by using {\ttfamily docker} with the available containers that contain the proper compilers, M\+PI, and all other dependencies. First, install {\ttfamily docker} on the system. On some systems, {\ttfamily docker-\/compose} might also need to be installed.

The {\ttfamily docker} builds are configured through {\ttfamily docker-\/compose} to use a shared, cached filesystem mount with the host for {\ttfamily ccache} to enable fast re-\/builds.

For {\ttfamily docker-\/compose}, the following variables can be set to configure the build. One may configure the architecture, compiler type (G\+NU, Clang, Intel, Nvidia) and compiler version, Linux distro (ubuntu or alpine), and distro version.

The default set of the docker configuration options is located in {\ttfamily vt/.env}, which {\ttfamily docker-\/compose} will read.


\begin{DoxyCode}
# Variables:
#   ARCH=\{amd64, arm64v8, ...\}
#   COMPILER\_TYPE=\{gnu, clang, intel, nvidia\}
#   COMPILER=\{gcc-7, gcc-8, gcc-9, gcc-10,
#             clang-4.0, clang-5.0, clang-6.0, clang-7, clang-8,
#             clang-9, clang-10,
#             icc-18, icc-19,
#             nvcc-10, nvcc-11, nvcc-11.2\}
#   REPO=lifflander1/vt
#   UBUNTU=\{18.04, 20.04\}
#   ULIMIT\_CORE=0
#
# DARMA/vt Configuration Variables:
#   VT\_LB=1                   # Enable load balancing
#   VT\_TRACE=0                # Enable tracing
#   VT\_MIMALLOC=0             # Enable mimalloc memory allocator
#   VT\_DOCS=0                 # Enable doxygen build
#   VT\_TRACE\_RT=0             # Enable tracing at runtime (for testing)
#   VT\_ASAN=0                 # Enable address sanitizer
#   VT\_UBSAN=0                # Enable undefined behavior sanitizer
#   VT\_WERROR=1               # Treat all warnings as errors
#   VT\_EXTENDED\_TESTS=1       # Build all the extended testing
#   VT\_ZOLTAN=0               # Build with Zoltan enabled
#   VT\_UNITY\_BUILD=0          # Build with Unity/Jumbo mode enabled
#   VT\_FCONTEXT=0             # Force use of fcontext for threading
#   VT\_DIAGNOSTICS=1          # Build with diagnostics enabled
#   VT\_DIAGNOSTICS\_RUNTIME=0  # Enable diagnostics at runtime by default
#   BUILD\_TYPE=release        # CMake build type
#   CODE\_COVERAGE=0           # Enable generation of code coverage reports
#   VT\_DEBUG\_VERBOSE          # Enable verbose debug prints at compile-time
#   VT\_NO\_COLOR\_ENABLED=0     # Set --vt\_no\_color flag to true by default
#   VT\_BUILD\_SHARED\_LIBS=0    # Build VT as shared library
\end{DoxyCode}


With these set, one may run the following for a non-\/interactive build with ubuntu. Or, to speed up the build process, the base container can be pulled for many of the common configurations\+: {\ttfamily docker-\/compose pull ubuntu-\/cpp}.


\begin{DoxyCode}
$ cd vt
$ docker-compose run -e BUILD\_TYPE=debug -e VT\_TRACE=1 ubuntu-cpp
\end{DoxyCode}


Or, alternatively, run a non-\/interactive build with alpine\+:


\begin{DoxyCode}
$ cd vt
$ docker-compose run -e BUILD\_TYPE=debug -e VT\_TRACE=1 alpine-cpp
\end{DoxyCode}


For an interactive build with ubuntu, where one can build, debug, and run {\ttfamily valgrind}, etc\+:


\begin{DoxyCode}
$ cd vt
$ docker-compose run -e BUILD\_TYPE=debug -e VT\_TRACE=1 ubuntu-cpp-interactive
# /vt/ci/build\_cpp.sh /vt /build
# /vt/ci/test\_cpp.sh /vt /build
# /vt/ci/build\_vt\_sample.sh /vt /build
\end{DoxyCode}


The same call applies to alpine distro builds if you swap {\ttfamily ubuntu-\/cpp-\/interactive} for {\ttfamily alpine-\/cpp-\/interactive}.

For more detailed information on configuring the docker build, read the documentation in {\ttfamily vt/docker-\/compose.\+yml}.\hypertarget{vt-build_test-vt}{}\subsection{Testing}\label{vt-build_test-vt}
After {\itshape vt} is built successfully, one may invoke the tests several ways. One may run {\ttfamily make test} or {\ttfamily ninja test} (depending on the generator used) or {\ttfamily ctest}, to run all the tests. Alternatively, the tests can be run automatically from the CI script\+:


\begin{DoxyCode}
$ vt/ci/test\_cpp.sh <full-path-to-vt-source> <full-path-to-build-dir>
\end{DoxyCode}
 \hypertarget{tutorial}{}\section{Tutorial}\label{tutorial}
Tutorial of how to use {\bfseries {\itshape vt}}\hypertarget{tutorial_tutorial-init-finalize-vt}{}\subsection{Initializing/\+Finalizing}\label{tutorial_tutorial-init-finalize-vt}
To initialize {\bfseries {\itshape vt}} all you must do is invoke the {\ttfamily \hyperlink{namespacevt_a580ef23e9a224fc0907da1a6db308b5b}{vt\+::initialize}} call with the program arguments, which creates and initializes a new {\bfseries {\itshape vt}} runtime. The initialize call reads the {\bfseries {\itshape vt}} arguments and removes them, leaving the remaining for the program.


\begin{DoxyCode}
\textcolor{keywordtype}{int} main(\textcolor{keywordtype}{int} argc, \textcolor{keywordtype}{char}** argv) \{
  \hyperlink{namespacevt_a580ef23e9a224fc0907da1a6db308b5b}{vt::initialize}(argc, argv);
   \textcolor{comment}{// program here}
  \hyperlink{namespacevt_a540d90dbd6e97b69f1dcbc9ee9314cff}{vt::finalize}();
\}
\end{DoxyCode}


If you are already using M\+PI in your program already, you don\textquotesingle{}t want to call initialize like this because {\bfseries {\itshape vt}} will initialize M\+PI itself. Instead, you should pass a pointer to the live M\+PI communicator to {\bfseries {\itshape vt}} for it to clone, as so\+:


\begin{DoxyCode}
\textcolor{keywordtype}{int} main(\textcolor{keywordtype}{int} argc, \textcolor{keywordtype}{char}** argv) \{
  MPI\_Init(&argc, &argv);
  \hyperlink{namespacevt_a580ef23e9a224fc0907da1a6db308b5b}{vt::initialize}(argc, argv, &MPI\_COMM\_WORLD);
   \textcolor{comment}{// program here}
  \hyperlink{namespacevt_a540d90dbd6e97b69f1dcbc9ee9314cff}{vt::finalize}();
  MPI\_Finalize();
\}
\end{DoxyCode}


If, for any reason, you want to predefine configuration, you can do it by creating {\ttfamily App\+Config} object, setting its members as you wish, and passing it to {\ttfamily \hyperlink{namespacevt_a580ef23e9a224fc0907da1a6db308b5b}{vt\+::initialize}}\+:


\begin{DoxyCode}
\textcolor{keywordtype}{int} main(\textcolor{keywordtype}{int} argc, \textcolor{keywordtype}{char}** argv) \{
  arguments::AppConfig appConfig\{\};
  appConfig.vt\_lb\_name = \textcolor{stringliteral}{"RotateLB"};
  appConfig.vt\_lb\_data = \textcolor{keyword}{true};

  \hyperlink{namespacevt_a580ef23e9a224fc0907da1a6db308b5b}{vt::initialize}(argc, argv, &appConfig);
   \textcolor{comment}{// program here}
  \hyperlink{namespacevt_a540d90dbd6e97b69f1dcbc9ee9314cff}{vt::finalize}();
\}
\end{DoxyCode}


You can do also do it if you initialized M\+PI on your own\+:


\begin{DoxyCode}
\textcolor{keywordtype}{int} main(\textcolor{keywordtype}{int} argc, \textcolor{keywordtype}{char}** argv) \{
  MPI\_Init(&argc, &argv);

  arguments::AppConfig appConfig\{\};
  appConfig.vt\_lb\_name = \textcolor{stringliteral}{"RotateLB"};
  appConfig.vt\_lb\_data = \textcolor{keyword}{true};

  \hyperlink{namespacevt_a580ef23e9a224fc0907da1a6db308b5b}{vt::initialize}(argc, argv, &MPI\_COMM\_WORLD, &appConfig);
   \textcolor{comment}{// program here}
  \hyperlink{namespacevt_a540d90dbd6e97b69f1dcbc9ee9314cff}{vt::finalize}();
  MPI\_Finalize();
\}
\end{DoxyCode}


It is worth noting that if you run your application with any of vt\textquotesingle{}s command-\/line arguments and at the same time you define and pass {\ttfamily App\+Config} to {\ttfamily \hyperlink{namespacevt_a580ef23e9a224fc0907da1a6db308b5b}{vt\+::initialize}}, C\+LI arguments have a higher priority. In other words, if you predefine in source code and give from the command line the same vt\textquotesingle{}s argument, but with a different value, the program will use the C\+LI one.

There is also an option to use configuration file. Refer to C\+L\+I11 documentation for details \href{https://cliutils.github.io/CLI11/book/chapters/config.html}{\tt https\+://cliutils.\+github.\+io/\+C\+L\+I11/book/chapters/config.\+html}. Important thing to remember -\/ C\+L\+I11 processes configuration file before command line arguments, so in the end command line arguments might overwrite values defined in configuration file.\hypertarget{tutorial_tutorial-walkthrough}{}\subsection{Tutorial Code Snippets}\label{tutorial_tutorial-walkthrough}
This page walks through the tutorial that exists in the source code. See {\ttfamily vt/tutorial/$\ast$.h} for the tutorial pieces in the repository. The main tutorial example code that stitches the pieces together is located in example\+: {\ttfamily vt/tutorial/tutorial\+\_\+main.\+h}. The actual code that compiles into an example is in {\ttfamily vt/examples/tutorial.\+cc}.


\begin{DoxyItemize}
\item Tutorial 1---Basic Concepts
\begin{DoxyItemize}
\item 1a\+: \hyperlink{tutorial-1a}{Learning about context}
\item 1b\+: \hyperlink{tutorial-1b}{Learning about active handlers}
\item 1c\+: \hyperlink{tutorial-1c}{Learning about serialization}
\item 1d\+: \hyperlink{tutorial-1d}{Learning about broadcasts}
\item 1e\+: \hyperlink{tutorial-1e}{Learning about rooted groups}
\item 1f\+: \hyperlink{tutorial-1f}{Learning about collective groups}
\item 1g\+: \hyperlink{tutorial-1g}{Learning about callbacks}
\item 1h\+: \hyperlink{tutorial-1h}{Learning about reductions}
\end{DoxyItemize}
\item Tutorial 2---Collections
\begin{DoxyItemize}
\item 2a\+: \hyperlink{tutorial-2a}{Learning about collections}
\item 2b\+: \hyperlink{tutorial-2b}{Learning about collection reductions}
\end{DoxyItemize}
\item Tutorial 3---Epochs
\begin{DoxyItemize}
\item 3a\+: \hyperlink{tutorial-3a}{Learning about collective epochs} 
\end{DoxyItemize}
\end{DoxyItemize}\hypertarget{tutorial-1a}{}\subsection{Learning about context}\label{tutorial-1a}

\begin{DoxyCodeInclude}
\textcolor{keyword}{static} \textcolor{keyword}{inline} \textcolor{keywordtype}{void} \hyperlink{namespacevt_1_1config_a6bd1d6215bda0d8ca02811798399f689a5dad3be974af03e75e234cb475c2cb20}{context}() \{
  \textcolor{comment}{/*}
\textcolor{comment}{   *  ::vt::theContext() can be used to obtain the node and number of nodes:}
\textcolor{comment}{   *  these correlate exactly to the MPI ranks and MPI size.}
\textcolor{comment}{   *}
\textcolor{comment}{   *  By default if you initialize VT without passing an MPI communication, VT}
\textcolor{comment}{   *  will initialize MPI and use an internal communicator. If you want VT to}
\textcolor{comment}{   *  interoperate, you can pass a communicator to the initialize invocation.}
\textcolor{comment}{   *}
\textcolor{comment}{   */}

  \textcolor{comment}{// Equivalent to: MPI\_Comm\_rank(...)}
  \hyperlink{namespacevt_a866da9d0efc19c0a1ce79e9e492f47e2}{NodeType} \textcolor{keyword}{const} this\_node = \hyperlink{namespacevt_a26551fe0e6e6a1371111df5b12c7e92c}{::vt::theContext}()->\hyperlink{structvt_1_1ctx_1_1_context_a0d52c263ce8516546a67443d9a86fa5f}{getNode}();

  \textcolor{comment}{// Equivalent to: MPI\_Comm\_size(...)}
  \hyperlink{namespacevt_a866da9d0efc19c0a1ce79e9e492f47e2}{NodeType} \textcolor{keyword}{const} num\_nodes = \hyperlink{namespacevt_a26551fe0e6e6a1371111df5b12c7e92c}{::vt::theContext}()->
      \hyperlink{structvt_1_1ctx_1_1_context_a7f41071aadf6d5fa9e1b6c703c5ff19d}{getNumNodes}();

  \textcolor{comment}{// The header-only library fmt is used for printing throughout VT. You can use}
  \textcolor{comment}{// it because the headers are included by default}
  ::fmt::print(\textcolor{stringliteral}{"this\_node=\{\}, num\_nodes=\{\}\(\backslash\)n"}, this\_node, num\_nodes);
\}
\end{DoxyCodeInclude}
\hypertarget{tutorial-1b}{}\subsection{Learning about active handlers}\label{tutorial-1b}

\begin{DoxyCodeInclude}
\textcolor{comment}{/*}
\textcolor{comment}{ * This is a user-defined message that can be sent to a node via VT's active}
\textcolor{comment}{ * message interface. A message in VT must be derived from the type}
\textcolor{comment}{ * vt::Message. The derived class will include the ``envelope'', which includes}
\textcolor{comment}{ * the handler, and other information for processing the message}
\textcolor{comment}{ */}

\textcolor{comment}{//            VT Base Message}
\textcolor{comment}{//           \(\backslash\)----------------/}
\textcolor{comment}{//            \(\backslash\)              /}
\textcolor{keyword}{struct }MyMsg : \hyperlink{structvt_1_1messaging_1_1_active_msg}{::vt::Message} \{
  \textcolor{comment}{// In general, a default constructor is required for the a message because it}
  \textcolor{comment}{// may be reconstructed by VT}
  MyMsg() = \textcolor{keywordflow}{default};

  \textcolor{comment}{// A normal constructor}
  MyMsg(\textcolor{keywordtype}{int} in\_a, \textcolor{keywordtype}{int} in\_b) : a\_(in\_a), b\_(in\_b) \{ \}

  \textcolor{keywordtype}{int} getA()\textcolor{keyword}{ const }\{ \textcolor{keywordflow}{return} a\_; \}
  \textcolor{keywordtype}{int} getB()\textcolor{keyword}{ const }\{ \textcolor{keywordflow}{return} b\_; \}

\textcolor{keyword}{private}:
  \textcolor{keywordtype}{int} a\_ = 0, b\_ = 0;
\};

\textcolor{comment}{/*}
\textcolor{comment}{ * Following are two active message handler declarations. These are the}
\textcolor{comment}{ * functions that can be remotely invoked by sending a message in VT. VT allows}
\textcolor{comment}{ * for multiple styles for handlers. The first (msgHandlerA) is the C-style}
\textcolor{comment}{ * active message handler. It is exactly equivalent (in terms of practical use}
\textcolor{comment}{ * ramifications) to the functor style, which follows (MsgHandlerB). Either}
\textcolor{comment}{ * style can be used to define a message handler. The only slight benefit the}
\textcolor{comment}{ * functor has is uniqueness in type, when enables introspection of the message}
\textcolor{comment}{ * type. This is realized in the sending side code: the function style requires}
\textcolor{comment}{ * the message type before the value function template parameter. However, both}
\textcolor{comment}{ * are *fully type checked* and will not allow the wrong message type to be}
\textcolor{comment}{ * sent.}
\textcolor{comment}{ */}

\textcolor{comment}{// Forward declaration for the active message handler (function ptr style)}
\textcolor{keyword}{static} \textcolor{keywordtype}{void} msgHandlerA(MyMsg* msg);

\textcolor{comment}{// Forward declaration for the active message handler (functor style)}
\textcolor{keyword}{struct }MsgHandlerB \{
  \textcolor{keywordtype}{void} operator()(MyMsg* msg);
\};

\textcolor{keyword}{static} \textcolor{keyword}{inline} \textcolor{keywordtype}{void} activeMessageNode() \{
  \hyperlink{namespacevt_a866da9d0efc19c0a1ce79e9e492f47e2}{NodeType} \textcolor{keyword}{const} this\_node = \hyperlink{namespacevt_a26551fe0e6e6a1371111df5b12c7e92c}{::vt::theContext}()->\hyperlink{structvt_1_1ctx_1_1_context_a0d52c263ce8516546a67443d9a86fa5f}{getNode}();
  \hyperlink{namespacevt_a866da9d0efc19c0a1ce79e9e492f47e2}{NodeType} \textcolor{keyword}{const} num\_nodes = \hyperlink{namespacevt_a26551fe0e6e6a1371111df5b12c7e92c}{::vt::theContext}()->
      \hyperlink{structvt_1_1ctx_1_1_context_a7f41071aadf6d5fa9e1b6c703c5ff19d}{getNumNodes}();
  (void)num\_nodes;  \textcolor{comment}{// don't warn about unused variable}

  \textcolor{comment}{/*}
\textcolor{comment}{   * A basic active message send essentially does an}
\textcolor{comment}{   * ``MPI\_Send(..,destination,...)'' to the destination node passed to}
\textcolor{comment}{   * ::vt::theMsg()->sendMsg(destination). The handler, which is passed as the}
\textcolor{comment}{   * second template argument, is the function that is triggered when the}
\textcolor{comment}{   * message arrives on the destination node.}
\textcolor{comment}{   *}
\textcolor{comment}{   * The theMsg()->sendMsg(..) does not serialize the message sent to the}
\textcolor{comment}{   * destination node. Even if the message has a serialize method (the above}
\textcolor{comment}{   * example does not), it is sent as bytes. This is because sendMsg always just}
\textcolor{comment}{   * sends the data directly that is passed to it.}
\textcolor{comment}{   *}
\textcolor{comment}{   * This example uses the function pointer style compared to the send inside of}
\textcolor{comment}{   * msgHandlerA, which uses the functor style send.}
\textcolor{comment}{   */}

  \textcolor{keywordflow}{if} (this\_node == 0) \{
    \hyperlink{namespacevt_a866da9d0efc19c0a1ce79e9e492f47e2}{NodeType} \textcolor{keyword}{const} to\_node = 1;
    \textcolor{keyword}{auto} msg = ::vt::makeMessage<MyMsg>(29,32);
    \hyperlink{namespacevt_aeafd31f866aeb4dc6fc2f6ee97136350}{::vt::theMsg}()->\hyperlink{group__preregister_ga0162a39473e7f9b490a79a7983d949ac}{sendMsg}<msgHandlerA>(to\_node, msg);
  \}
\}

\textcolor{keyword}{static} \textcolor{keywordtype}{void} msgHandlerA(MyMsg* msg) \{
  \textcolor{comment}{/*}
\textcolor{comment}{   * VT provides vtAssert (and a half-dozen variants) that replace the simple}
\textcolor{comment}{   * the assert call found in cassert.}
\textcolor{comment}{   */}
  \hyperlink{config__assert_8h_aeddd4990a496e91a0ca5d6c16437359b}{vtAssert}(msg->getA() == 29, \textcolor{stringliteral}{"Value a incorrect"});
  \hyperlink{config__assert_8h_aeddd4990a496e91a0ca5d6c16437359b}{vtAssert}(msg->getB() == 32, \textcolor{stringliteral}{"Value b incorrect"});

  \textcolor{keyword}{auto} \textcolor{keyword}{const} cur\_node = \hyperlink{namespacevt_a26551fe0e6e6a1371111df5b12c7e92c}{::vt::theContext}()->\hyperlink{structvt_1_1ctx_1_1_context_a0d52c263ce8516546a67443d9a86fa5f}{getNode}();

  ::fmt::print(\textcolor{stringliteral}{"msgHandlerA: triggered on node=\{\}\(\backslash\)n"}, cur\_node);

  \textcolor{comment}{/* Node 0 sends MyMsg to node 1 in the above code so this should execute on}
\textcolor{comment}{   * node 1 */}
  \hyperlink{config__assert_8h_aeddd4990a496e91a0ca5d6c16437359b}{vtAssert}(cur\_node == 1, \textcolor{stringliteral}{"This handler should execute on node 1"});

  \textcolor{comment}{/*}
\textcolor{comment}{   * In response to this message, node 1 sends a message back to node 0. This}
\textcolor{comment}{   * invocation uses the functor style send.}
\textcolor{comment}{   *}
\textcolor{comment}{   * -- Message Ownsership --}
\textcolor{comment}{   * When an API that 'sends a message' is called, ownership is relinquished.}
\textcolor{comment}{   *}
\textcolor{comment}{   * The loss of ownership is the same as using std::move and the MsgPtr object}
\textcolor{comment}{   * supplied becomes invalid for later use. This use includes attempting}
\textcolor{comment}{   * to send the same message twice. std::move can be used explicit if desired.}
\textcolor{comment}{   *}
\textcolor{comment}{   * A vtAssert will be raised if such an invalid use is detected.}
\textcolor{comment}{   */}
  \hyperlink{namespacevt_a866da9d0efc19c0a1ce79e9e492f47e2}{NodeType} \textcolor{keyword}{const} to\_node = 0;
  \textcolor{keyword}{auto} msg2 = ::vt::makeMessage<MyMsg>(10,20);

  \hyperlink{namespacevt_aeafd31f866aeb4dc6fc2f6ee97136350}{::vt::theMsg}()->\hyperlink{group__preregister_ga0162a39473e7f9b490a79a7983d949ac}{sendMsg}<MsgHandlerB>(to\_node, msg2);

  \textcolor{comment}{// Alternate/equivalent form with explicit (and optional) std::move:}
  \textcolor{comment}{//}
  \textcolor{comment}{//   ::vt::theMsg()->sendMsg<MsgHandlerB, MyMsg>(to\_node, std::move(msg2));}
  \textcolor{comment}{//}
  \textcolor{comment}{// In both cases it is invalid to attempt to use msg2 here as the ownership}
  \textcolor{comment}{// of the message has been relinquished/lost when making the call.}
\}

\textcolor{keywordtype}{void} MsgHandlerB::operator()(MyMsg* msg) \{
  \textcolor{keyword}{auto} \textcolor{keyword}{const} cur\_node = \hyperlink{namespacevt_a26551fe0e6e6a1371111df5b12c7e92c}{::vt::theContext}()->\hyperlink{structvt_1_1ctx_1_1_context_a0d52c263ce8516546a67443d9a86fa5f}{getNode}();
  \hyperlink{config__assert_8h_aeddd4990a496e91a0ca5d6c16437359b}{vtAssert}(msg->getA() == 10, \textcolor{stringliteral}{"Value a incorrect"});
  \hyperlink{config__assert_8h_aeddd4990a496e91a0ca5d6c16437359b}{vtAssert}(msg->getB() == 20, \textcolor{stringliteral}{"Value b incorrect"});
  \hyperlink{config__assert_8h_aeddd4990a496e91a0ca5d6c16437359b}{vtAssert}(cur\_node == 0, \textcolor{stringliteral}{"This handler should execute on node 0"});
  ::fmt::print(\textcolor{stringliteral}{"msgHandlerB: triggered on node=\{\}\(\backslash\)n"}, cur\_node);
\}
\end{DoxyCodeInclude}
\hypertarget{tutorial-1c}{}\subsection{Learning about serialization}\label{tutorial-1c}

\begin{DoxyCodeInclude}
\textcolor{comment}{/*}
\textcolor{comment}{ * Some user defined data structures}
\textcolor{comment}{ */}
\textcolor{keyword}{struct }Particle \{
  Particle() = \textcolor{keywordflow}{default};
  Particle(\textcolor{keywordtype}{double} in\_x, \textcolor{keywordtype}{double} in\_y, \textcolor{keywordtype}{double} in\_z)
    : x(in\_x), y(in\_y), z(in\_z)
  \{ \}

  \textcolor{comment}{// Non-message types can use serializers like this.}
  \textcolor{keyword}{template} <\textcolor{keyword}{typename} SerializerT>
  \textcolor{keywordtype}{void} \hyperlink{namespacecheckpoint_a075da4e7344cf037943362517e606c3a}{serialize}(SerializerT& s) \{
    s | x | y | z;
  \}

  \textcolor{keywordtype}{double} x, y, z;
\};

\textcolor{keyword}{struct }Data \{
  \textcolor{keywordtype}{int} a;

  \textcolor{keyword}{template} <\textcolor{keyword}{typename} SerializerT>
  \textcolor{keywordtype}{void} \hyperlink{namespacecheckpoint_a075da4e7344cf037943362517e606c3a}{serialize}(SerializerT& s) \{
    s | a;
  \}
\};

\textcolor{comment}{// TODO - show data with non-intrusive serialization}
\textcolor{comment}{// TODO - show data that does not need serialization}

\textcolor{comment}{//                  VT Base Message}
\textcolor{comment}{//                 \(\backslash\)----------------/}
\textcolor{comment}{//                  \(\backslash\)              /}
\textcolor{keyword}{struct }ParticleMsg : \hyperlink{structvt_1_1messaging_1_1_active_msg}{::vt::Message} \{
  \textcolor{keyword}{using} MessageParentType = \hyperlink{namespacevt_a3a3ddfef40b4c90915fa43cdd5f129ea}{::vt::Message}; \textcolor{comment}{// base message}
  \hyperlink{message__serialize_8h_a54128d5338aaa9c918179826085e13d1}{vt\_msg\_serialize\_required}();         \textcolor{comment}{// mark serialization mode}

  ParticleMsg() = \textcolor{keywordflow}{default};

  ParticleMsg(\textcolor{keywordtype}{int} in\_x, \textcolor{keywordtype}{int} in\_y, \textcolor{keywordtype}{int} in\_z)
    : x(in\_x), y(in\_y), z(in\_z)
  \{ \}

  \textcolor{comment}{/*}
\textcolor{comment}{   * Implement a serialize method so the std::vector and pointer are properly}
\textcolor{comment}{   * serialization on the send.}
\textcolor{comment}{   */}
  \textcolor{keyword}{template} <\textcolor{keyword}{typename} SerializerT>
  \textcolor{keywordtype}{void} \hyperlink{namespacecheckpoint_a075da4e7344cf037943362517e606c3a}{serialize}(SerializerT& s) \{
    \hyperlink{namespacecheckpoint_a075da4e7344cf037943362517e606c3a}{MessageParentType::serialize}(s);    \textcolor{comment}{// ensure parent is serialized
       consistently}

    s | particles;
    s | x | y | z;

    \textcolor{comment}{// Serialize a pointer by first serializing whether it is non-null and then}
    \textcolor{comment}{// the actual value if it's not non-null}
    \textcolor{keywordtype}{bool} has\_a = a != \textcolor{keyword}{nullptr};
    s | has\_a;
    \textcolor{keywordflow}{if} (has\_a) \{
      s | *a;
    \}
  \}

\textcolor{keyword}{public}:
  std::vector<Particle> particles;
  \textcolor{keywordtype}{int} x = 0, y = 0, z = 0;
  Data* a = \textcolor{keyword}{nullptr};
\};

\textcolor{comment}{// Forward declaration for the active message handler}
\textcolor{keyword}{static} \textcolor{keywordtype}{void} msgSerialA(ParticleMsg* msg);

\textcolor{comment}{// Tutorial code to demonstrate serialization in active message sends}
\textcolor{keyword}{static} \textcolor{keyword}{inline} \textcolor{keywordtype}{void} activeMessageSerialization() \{
  \hyperlink{namespacevt_a866da9d0efc19c0a1ce79e9e492f47e2}{NodeType} \textcolor{keyword}{const} this\_node = \hyperlink{namespacevt_a26551fe0e6e6a1371111df5b12c7e92c}{::vt::theContext}()->\hyperlink{structvt_1_1ctx_1_1_context_a0d52c263ce8516546a67443d9a86fa5f}{getNode}();
  \hyperlink{namespacevt_a866da9d0efc19c0a1ce79e9e492f47e2}{NodeType} \textcolor{keyword}{const} num\_nodes = \hyperlink{namespacevt_a26551fe0e6e6a1371111df5b12c7e92c}{::vt::theContext}()->
      \hyperlink{structvt_1_1ctx_1_1_context_a7f41071aadf6d5fa9e1b6c703c5ff19d}{getNumNodes}();
  (void)num\_nodes;  \textcolor{comment}{// don't warn about unused variable}

  \textcolor{comment}{/*}
\textcolor{comment}{   * The theMsg()->sendMsg(..) will serialize the message sent to the}
\textcolor{comment}{   * destination node if it has a serialize method. If not, it will send the}
\textcolor{comment}{   * message as if it is sent directly the sendMsg.}
\textcolor{comment}{   */}

  \textcolor{keywordflow}{if} (this\_node == 0) \{
    \hyperlink{namespacevt_a866da9d0efc19c0a1ce79e9e492f47e2}{NodeType} \textcolor{keyword}{const} to\_node = 1;
    \textcolor{keyword}{auto} msg = ::vt::makeMessage<ParticleMsg>(1,2,3);
    msg->particles.push\_back(Particle\{10,11,12\});
    msg->particles.push\_back(Particle\{13,14,15\});
    \hyperlink{namespacevt_aeafd31f866aeb4dc6fc2f6ee97136350}{::vt::theMsg}()->\hyperlink{group__preregister_ga0162a39473e7f9b490a79a7983d949ac}{sendMsg}<msgSerialA>(to\_node, msg);
  \}
\}

\textcolor{comment}{// Message handler}
\textcolor{keyword}{static} \textcolor{keywordtype}{void} msgSerialA(ParticleMsg* msg) \{
  \hyperlink{config__assert_8h_aeddd4990a496e91a0ca5d6c16437359b}{vtAssert}(msg->particles.size() == 2, \textcolor{stringliteral}{"Should be two particles"});
  \hyperlink{config__assert_8h_aeddd4990a496e91a0ca5d6c16437359b}{vtAssert}(msg->x == 1 && msg->y == 2 && msg->z == 3, \textcolor{stringliteral}{"Values x,y,z incorrect"});

  \textcolor{keyword}{auto} \textcolor{keyword}{const} cur\_node = \hyperlink{namespacevt_a26551fe0e6e6a1371111df5b12c7e92c}{::vt::theContext}()->\hyperlink{structvt_1_1ctx_1_1_context_a0d52c263ce8516546a67443d9a86fa5f}{getNode}();

  \textcolor{comment}{/* Node 0 sends MyMsg to node 1 in the above code so this should execute on}
\textcolor{comment}{   * node 1 */}
  \hyperlink{config__assert_8h_aeddd4990a496e91a0ca5d6c16437359b}{vtAssert}(cur\_node == 1, \textcolor{stringliteral}{"This handler should execute on node 1"});

  ::fmt::print(\textcolor{stringliteral}{"msgSerialA: triggered on node=\{\}\(\backslash\)n"}, cur\_node);
\}
\end{DoxyCodeInclude}
\hypertarget{tutorial-1d}{}\subsection{Learning about broadcasts}\label{tutorial-1d}

\begin{DoxyCodeInclude}
\textcolor{comment}{//                  VT Base Message}
\textcolor{comment}{//                 \(\backslash\)----------------/}
\textcolor{comment}{//                  \(\backslash\)              /}
\textcolor{keyword}{struct }MyDataMsg : \hyperlink{structvt_1_1messaging_1_1_active_msg}{::vt::Message} \{
  MyDataMsg() = \textcolor{keywordflow}{default};
  MyDataMsg(\textcolor{keywordtype}{double} in\_x, \textcolor{keywordtype}{double} in\_y, \textcolor{keywordtype}{double} in\_z)
    : x(in\_x), y(in\_y), z(in\_z)
  \{ \}

  \textcolor{keywordtype}{double} getX()\textcolor{keyword}{ const }\{ \textcolor{keywordflow}{return} x; \}
  \textcolor{keywordtype}{double} getY()\textcolor{keyword}{ const }\{ \textcolor{keywordflow}{return} y; \}
  \textcolor{keywordtype}{double} getZ()\textcolor{keyword}{ const }\{ \textcolor{keywordflow}{return} z; \}

\textcolor{keyword}{private}:
  \textcolor{keywordtype}{double} x = 0.0, y = 0.0, z = 0.0;
\};

\textcolor{comment}{// Forward declaration for the active message handler}
\textcolor{keyword}{static} \textcolor{keywordtype}{void} msgHandlerX(MyDataMsg* msg);

\textcolor{comment}{// Tutorial code to demonstrate broadcasting a message to the entire system}
\textcolor{keyword}{static} \textcolor{keyword}{inline} \textcolor{keywordtype}{void} activeMessageBroadcast() \{
  \hyperlink{namespacevt_a866da9d0efc19c0a1ce79e9e492f47e2}{NodeType} \textcolor{keyword}{const} this\_node = \hyperlink{namespacevt_a26551fe0e6e6a1371111df5b12c7e92c}{::vt::theContext}()->\hyperlink{structvt_1_1ctx_1_1_context_a0d52c263ce8516546a67443d9a86fa5f}{getNode}();
  \hyperlink{namespacevt_a866da9d0efc19c0a1ce79e9e492f47e2}{NodeType} \textcolor{keyword}{const} num\_nodes = \hyperlink{namespacevt_a26551fe0e6e6a1371111df5b12c7e92c}{::vt::theContext}()->
      \hyperlink{structvt_1_1ctx_1_1_context_a7f41071aadf6d5fa9e1b6c703c5ff19d}{getNumNodes}();
  (void)num\_nodes;  \textcolor{comment}{// don't warn about unused variable}

  \textcolor{comment}{/*}
\textcolor{comment}{   * default\_proxy.broadcast(..) will send the message to every node in the}
\textcolor{comment}{   * system. Every node will include all the nodes that VT has depending on the}
\textcolor{comment}{   * MPI communicator passed in or the size attained (number of ranks) when}
\textcolor{comment}{   * executing MPI init directly in non-interoperability mode.}
\textcolor{comment}{   *}
\textcolor{comment}{   * -- Message Ownership --}
\textcolor{comment}{   * default\_proxy.broadcast(..) will create message out of args passed}
\textcolor{comment}{   * into it. There's an alternative - default\_proxy.broadcastMsg(..) - which}
\textcolor{comment}{   * relinquishes ownership of the message passed to:}
\textcolor{comment}{   *}
\textcolor{comment}{   *   auto msg = ::vt::makeMessage<MyDataMsg>(1.0, 2.0, 3.0);}
\textcolor{comment}{   *   auto const default\_proxy = theObjGroup()->getDefault();}
\textcolor{comment}{   *   default\_proxy.broadcastMsg<MyDataMsg, msgHandlerX>(msg);}
\textcolor{comment}{   *}
\textcolor{comment}{   *  Most calls to VT that supply a message are expected to relinquish ownership.}
\textcolor{comment}{   */}

  \textcolor{keywordflow}{if} (this\_node == 0) \{
    \textcolor{keyword}{auto} \textcolor{keyword}{const} default\_proxy = \hyperlink{namespacevt_a833f0115b692f578167cbd88e30d39c5}{theObjGroup}()->\hyperlink{structvt_1_1objgroup_1_1_obj_group_manager_a4447b8f4881710bc7b0b7a37de9d10b9}{getDefault}();
    default\_proxy.\hyperlink{structvt_1_1objgroup_1_1proxy_1_1_proxy_3_01void_01_4_a13f8d6751a246a449e7d13ae7eed3541}{broadcast}<MyDataMsg, msgHandlerX>(1.0, 2.0, 3.0);
  \}
\}

\textcolor{comment}{// Message handler}
\textcolor{keyword}{static} \textcolor{keywordtype}{void} msgHandlerX(MyDataMsg* msg) \{
  \hyperlink{config__assert_8h_aeddd4990a496e91a0ca5d6c16437359b}{vtAssert}(
    msg->getX() == 1.0 && msg->getY() == 2.0 && msg->getZ() == 3.0,
    \textcolor{stringliteral}{"Values x,y,z incorrect"}
  );

  \textcolor{keyword}{auto} \textcolor{keyword}{const} cur\_node = \hyperlink{namespacevt_a26551fe0e6e6a1371111df5b12c7e92c}{::vt::theContext}()->\hyperlink{structvt_1_1ctx_1_1_context_a0d52c263ce8516546a67443d9a86fa5f}{getNode}();
  ::fmt::print(\textcolor{stringliteral}{"msgHandlerX: triggered on node=\{\}\(\backslash\)n"}, cur\_node);
\}
\end{DoxyCodeInclude}
\hypertarget{tutorial-1e}{}\subsection{Learning about rooted groups}\label{tutorial-1e}

\begin{DoxyCodeInclude}
\textcolor{comment}{//                  VT Base Message}
\textcolor{comment}{//                 \(\backslash\)----------------/}
\textcolor{comment}{//                  \(\backslash\)              /}
\textcolor{keyword}{struct }MySimpleMsg : \hyperlink{structvt_1_1messaging_1_1_active_msg}{::vt::Message} \{ \};

\textcolor{comment}{// Forward declaration for the active message handler}
\textcolor{keyword}{static} \textcolor{keywordtype}{void} msgHandlerGroupA(MySimpleMsg* msg);

\textcolor{comment}{// Tutorial code to demonstrate rooted group creation and broadcast to that group}
\textcolor{keyword}{static} \textcolor{keyword}{inline} \textcolor{keywordtype}{void} activeMessageGroupRoot() \{
  \hyperlink{namespacevt_a866da9d0efc19c0a1ce79e9e492f47e2}{NodeType} \textcolor{keyword}{const} this\_node = \hyperlink{namespacevt_a26551fe0e6e6a1371111df5b12c7e92c}{::vt::theContext}()->\hyperlink{structvt_1_1ctx_1_1_context_a0d52c263ce8516546a67443d9a86fa5f}{getNode}();
  \hyperlink{namespacevt_a866da9d0efc19c0a1ce79e9e492f47e2}{NodeType} \textcolor{keyword}{const} num\_nodes = \hyperlink{namespacevt_a26551fe0e6e6a1371111df5b12c7e92c}{::vt::theContext}()->
      \hyperlink{structvt_1_1ctx_1_1_context_a7f41071aadf6d5fa9e1b6c703c5ff19d}{getNumNodes}();

  \textcolor{comment}{/*}
\textcolor{comment}{   * This is an example of the rooted group creation and broadcast to that}
\textcolor{comment}{   * group. A group allows the user to create a subset of nodes. A broadcast by}
\textcolor{comment}{   * default sends the message to every node in the default group (which}
\textcolor{comment}{   * includes all nodes). If a explicit group is set in the envelope, the}
\textcolor{comment}{   * broadcast will only arrive on the nodes in that group.}
\textcolor{comment}{   */}

  \textcolor{keywordflow}{if} (this\_node == 0) \{
    \textcolor{comment}{// Create range for the group [1,num\_nodes);}
    \textcolor{keyword}{auto} range = std::make\_unique<::vt::group::region::Range>(1, num\_nodes);
    \textcolor{comment}{// The non-collective group is created by a single node based on a range or}
    \textcolor{comment}{// list of nodes. The lambda is executed once the group is created. By}
    \textcolor{comment}{// setting the group in the envelope of the message and broadcasting the}
    \textcolor{comment}{// message will arrive on the set of nodes included in the group}
    \textcolor{keyword}{auto} \textcolor{keywordtype}{id} = \hyperlink{namespacevt_a4548473dce44fb654400009e2b2fe64e}{theGroup}()->\hyperlink{structvt_1_1group_1_1_group_manager_a076330445139abce77e1f6ab6e4358c5}{newGroup}(std::move(range), [](
      \hyperlink{namespacevt_a27b5e4411c9b6140c49100e050e2f743}{GroupType} group\_id)\{
      fmt::print(\textcolor{stringliteral}{"Group is created: id=\{:x\}\(\backslash\)n"}, group\_id);
      \textcolor{keyword}{auto} msg = makeMessage<MySimpleMsg>();
      \hyperlink{namespacevt_a9f9d4ce6034c0eaaf98bdffd33d0e1c7}{envelopeSetGroup}(msg->env, group\_id);
      \hyperlink{namespacevt_aeafd31f866aeb4dc6fc2f6ee97136350}{theMsg}()->\hyperlink{group__typesafehan_ga1a347aac6b6b8d47c9b1c8cc11b7f33e}{broadcastMsg}<msgHandlerGroupA>(msg);
    \});
    \textcolor{comment}{// The `id' that is returned from the newGroup invocation, can be used}
    \textcolor{comment}{// anywhere in the system to broadcast (multicast) to this group.}
    (void)\textcolor{keywordtype}{id};  \textcolor{comment}{// don't warn about unused variable}
  \}
\}

\textcolor{comment}{// Message handler}
\textcolor{keyword}{static} \textcolor{keywordtype}{void} msgHandlerGroupA(MySimpleMsg* msg) \{
  \textcolor{keyword}{auto} \textcolor{keyword}{const} cur\_node = \hyperlink{namespacevt_a26551fe0e6e6a1371111df5b12c7e92c}{::vt::theContext}()->\hyperlink{structvt_1_1ctx_1_1_context_a0d52c263ce8516546a67443d9a86fa5f}{getNode}();
  \hyperlink{config__assert_8h_aeddd4990a496e91a0ca5d6c16437359b}{vtAssert}(cur\_node >= 1, \textcolor{stringliteral}{"This handler should only execute on nodes >= 1"});

  ::fmt::print(\textcolor{stringliteral}{"msgHandlerGroupA: triggered on node=\{\}\(\backslash\)n"}, cur\_node);
\}
\end{DoxyCodeInclude}
\hypertarget{tutorial-1f}{}\subsection{Learning about collective groups}\label{tutorial-1f}

\begin{DoxyCodeInclude}
\textcolor{comment}{//                  VT Base Message}
\textcolor{comment}{//                 \(\backslash\)----------------/}
\textcolor{comment}{//                  \(\backslash\)              /}
\textcolor{keyword}{struct }MySimpleMsg2 : \hyperlink{structvt_1_1messaging_1_1_active_msg}{::vt::Message} \{ \};

\textcolor{comment}{// Forward declaration for the active message handler}
\textcolor{keyword}{static} \textcolor{keywordtype}{void} msgHandlerGroupB(MySimpleMsg2* msg);

\textcolor{comment}{// Tutorial code to demonstrate collective group creation and broadcast to that group}
\textcolor{keyword}{static} \textcolor{keyword}{inline} \textcolor{keywordtype}{void} activeMessageGroupCollective() \{
  \hyperlink{namespacevt_a866da9d0efc19c0a1ce79e9e492f47e2}{NodeType} \textcolor{keyword}{const} this\_node = \hyperlink{namespacevt_a26551fe0e6e6a1371111df5b12c7e92c}{::vt::theContext}()->\hyperlink{structvt_1_1ctx_1_1_context_a0d52c263ce8516546a67443d9a86fa5f}{getNode}();
  \hyperlink{namespacevt_a866da9d0efc19c0a1ce79e9e492f47e2}{NodeType} \textcolor{keyword}{const} num\_nodes = \hyperlink{namespacevt_a26551fe0e6e6a1371111df5b12c7e92c}{::vt::theContext}()->
      \hyperlink{structvt_1_1ctx_1_1_context_a7f41071aadf6d5fa9e1b6c703c5ff19d}{getNumNodes}();
  (void)num\_nodes;  \textcolor{comment}{// don't warn about unused variable}

  \textcolor{comment}{/*}
\textcolor{comment}{   * This is an example of the collective group creation and broadcast to that}
\textcolor{comment}{   * group. A group allows the user to create a subset of nodes. The collective}
\textcolor{comment}{   * group allows all nodes to participate in creating the group by passing a}
\textcolor{comment}{   * boolean that indicates if they are apart of the group.}
\textcolor{comment}{   *}
\textcolor{comment}{   * Unlike the rooted group creation (which requires an initial set at a root}
\textcolor{comment}{   * node), the collective group is fully distributed: its creation and}
\textcolor{comment}{   * storage. The set of all nodes included is never stored in a central}
\textcolor{comment}{   * location. This is managed by efficient distributed algorithms that create a}
\textcolor{comment}{   * spanning tree based on the filter and rebalance it depending on outcomes.}
\textcolor{comment}{   */}

  \textcolor{keyword}{auto} \textcolor{keyword}{const}& is\_even\_node = this\_node % 2 == 0;

  \textcolor{keyword}{auto} group = \hyperlink{namespacevt_a4548473dce44fb654400009e2b2fe64e}{theGroup}()->\hyperlink{structvt_1_1group_1_1_group_manager_a92b301d6cf77af7d57ed822c6f044d58}{newGroupCollective}(
    is\_even\_node, [](\hyperlink{namespacevt_a27b5e4411c9b6140c49100e050e2f743}{GroupType} group\_id)\{
      fmt::print(\textcolor{stringliteral}{"Group is created: id=\{:x\}\(\backslash\)n"}, group\_id);

      \textcolor{comment}{// In this example, node 1 broadcasts to the group of even nodes}
      \textcolor{keyword}{auto} \textcolor{keyword}{const} my\_node = \hyperlink{namespacevt_a26551fe0e6e6a1371111df5b12c7e92c}{::vt::theContext}()->\hyperlink{structvt_1_1ctx_1_1_context_a0d52c263ce8516546a67443d9a86fa5f}{getNode}();
      \textcolor{keywordflow}{if} (my\_node == 1) \{
        \textcolor{keyword}{auto} msg = makeMessage<MySimpleMsg2>();
        \hyperlink{namespacevt_a9f9d4ce6034c0eaaf98bdffd33d0e1c7}{envelopeSetGroup}(msg->env, group\_id);
        \hyperlink{namespacevt_aeafd31f866aeb4dc6fc2f6ee97136350}{theMsg}()->\hyperlink{group__typesafehan_ga1a347aac6b6b8d47c9b1c8cc11b7f33e}{broadcastMsg}<msgHandlerGroupB>(msg);
      \}
    \}
  );
  (void)group;  \textcolor{comment}{// don't warn about unused variable}
\}

\textcolor{comment}{// Message handler}
\textcolor{keyword}{static} \textcolor{keywordtype}{void} msgHandlerGroupB(MySimpleMsg2* msg) \{
  \textcolor{keyword}{auto} \textcolor{keyword}{const} cur\_node = \hyperlink{namespacevt_a26551fe0e6e6a1371111df5b12c7e92c}{::vt::theContext}()->\hyperlink{structvt_1_1ctx_1_1_context_a0d52c263ce8516546a67443d9a86fa5f}{getNode}();
  \hyperlink{config__assert_8h_aeddd4990a496e91a0ca5d6c16437359b}{vtAssert}(cur\_node % 2 == 0, \textcolor{stringliteral}{"This handler should only execute on even nodes"});

  ::fmt::print(\textcolor{stringliteral}{"msgHandlerGroupB: triggered on node=\{\}\(\backslash\)n"}, cur\_node);
\}
\end{DoxyCodeInclude}
\hypertarget{tutorial-1g}{}\subsection{Learning about callbacks}\label{tutorial-1g}

\begin{DoxyCodeInclude}
\textcolor{comment}{//              VT Base Message}
\textcolor{comment}{//             \(\backslash\)----------------/}
\textcolor{comment}{//              \(\backslash\)              /}
\textcolor{keyword}{struct }DataMsg : \hyperlink{structvt_1_1messaging_1_1_active_msg}{::vt::Message} \{ \};

\textcolor{keyword}{struct }MsgWithCallback : \hyperlink{structvt_1_1messaging_1_1_active_msg}{::vt::Message} \{
  MsgWithCallback() = \textcolor{keywordflow}{default};
  \textcolor{keyword}{explicit} MsgWithCallback(Callback<DataMsg> in\_cb) : cb(in\_cb) \{\}

  Callback<DataMsg> cb;
\};


\textcolor{comment}{// Forward declaration for the active message handler}
\textcolor{keyword}{static} \textcolor{keywordtype}{void} getCallbackHandler(MsgWithCallback* msg);

\textcolor{comment}{// An active message handler used as the target for a callback}
\textcolor{keyword}{static} \textcolor{keywordtype}{void} callbackHandler(DataMsg* msg) \{
  \hyperlink{namespacevt_a866da9d0efc19c0a1ce79e9e492f47e2}{NodeType} \textcolor{keyword}{const} cur\_node = \hyperlink{namespacevt_a26551fe0e6e6a1371111df5b12c7e92c}{::vt::theContext}()->\hyperlink{structvt_1_1ctx_1_1_context_a0d52c263ce8516546a67443d9a86fa5f}{getNode}();
  ::fmt::print(\textcolor{stringliteral}{"\{\}: triggering active message callback\(\backslash\)n"}, cur\_node);
\}

\textcolor{comment}{// An active message handler used as the target for a callback}
\textcolor{keyword}{static} \textcolor{keywordtype}{void} callbackBcastHandler(DataMsg* msg) \{
  \hyperlink{namespacevt_a866da9d0efc19c0a1ce79e9e492f47e2}{NodeType} \textcolor{keyword}{const} cur\_node = \hyperlink{namespacevt_a26551fe0e6e6a1371111df5b12c7e92c}{::vt::theContext}()->\hyperlink{structvt_1_1ctx_1_1_context_a0d52c263ce8516546a67443d9a86fa5f}{getNode}();
  ::fmt::print(\textcolor{stringliteral}{"\{\}: triggering active message callback bcast\(\backslash\)n"}, cur\_node);
\}

\textcolor{comment}{// A simple context object}
\textcolor{keyword}{struct }MyContext \{ \};
\textcolor{keyword}{static} MyContext \hyperlink{namespacevt_1_1config_a0551245b6b893932b95aaf8eac94eed1}{ctx} = \{\};

\textcolor{comment}{// A message handler with context used as the target for a callback}
\textcolor{keyword}{static} \textcolor{keywordtype}{void} callbackCtx(DataMsg* msg, MyContext* cbctx) \{
  \hyperlink{namespacevt_a866da9d0efc19c0a1ce79e9e492f47e2}{NodeType} \textcolor{keyword}{const} cur\_node = \hyperlink{namespacevt_a26551fe0e6e6a1371111df5b12c7e92c}{::vt::theContext}()->\hyperlink{structvt_1_1ctx_1_1_context_a0d52c263ce8516546a67443d9a86fa5f}{getNode}();
  ::fmt::print(\textcolor{stringliteral}{"\{\}: triggering context callback\(\backslash\)n"}, cur\_node);
\}


\textcolor{comment}{// Tutorial code to demonstrate using a callback}
\textcolor{keyword}{static} \textcolor{keyword}{inline} \textcolor{keywordtype}{void} activeMessageCallback() \{
  \hyperlink{namespacevt_a866da9d0efc19c0a1ce79e9e492f47e2}{NodeType} \textcolor{keyword}{const} this\_node = \hyperlink{namespacevt_a26551fe0e6e6a1371111df5b12c7e92c}{::vt::theContext}()->\hyperlink{structvt_1_1ctx_1_1_context_a0d52c263ce8516546a67443d9a86fa5f}{getNode}();
  \hyperlink{namespacevt_a866da9d0efc19c0a1ce79e9e492f47e2}{NodeType} \textcolor{keyword}{const} num\_nodes = \hyperlink{namespacevt_a26551fe0e6e6a1371111df5b12c7e92c}{::vt::theContext}()->
      \hyperlink{structvt_1_1ctx_1_1_context_a7f41071aadf6d5fa9e1b6c703c5ff19d}{getNumNodes}();
  (void)num\_nodes;  \textcolor{comment}{// don't warn about unused variable}

  \textcolor{comment}{/*}
\textcolor{comment}{   * Callbacks allow one to generalize the notion of an endpoint with a abstract}
\textcolor{comment}{   * interface the callee can use without changing code. A callback can trigger}
\textcolor{comment}{   * a lambda, handler on a node, handler broadcast, handler/lambda with a}
\textcolor{comment}{   * context, message send of virtual context collection (element or broadcast)}
\textcolor{comment}{   */}

  \textcolor{keywordflow}{if} (this\_node == 0) \{
    \textcolor{comment}{// Node sending the callback message to, which shall invoke the callback}
    \hyperlink{namespacevt_a866da9d0efc19c0a1ce79e9e492f47e2}{NodeType} \textcolor{keyword}{const} to\_node = 1;
    \textcolor{comment}{// Node that we want to callback to execute on}
    \hyperlink{namespacevt_a866da9d0efc19c0a1ce79e9e492f47e2}{NodeType} \textcolor{keyword}{const} cb\_node = 0;

    \textcolor{comment}{// Example lambda callback (void)}
    \textcolor{keyword}{auto} void\_fn = [=]\{
      ::fmt::print(\textcolor{stringliteral}{"\{\}: triggering void function callback\(\backslash\)n"}, this\_node);
    \};

    \textcolor{comment}{// Example of a void lambda callback}
    \{
      \textcolor{keyword}{auto} cb = \hyperlink{namespacevt_a673b109e94c7bca58313504c83e1da94}{::vt::theCB}()->\hyperlink{structvt_1_1pipe_1_1_pipe_manager_a9ceec59c887d0fa1498b931c788962f6}{makeFunc}(
      \hyperlink{namespacevt_1_1pipe_acb42b284378c0fdac1d7c6335dc26f58ae1a9dc9f23534e63de9df0d540ac1611}{vt::pipe::LifetimeEnum::Once}, void\_fn);
      \textcolor{keyword}{auto} msg = ::vt::makeMessage<MsgWithCallback>(cb);
      \hyperlink{namespacevt_aeafd31f866aeb4dc6fc2f6ee97136350}{::vt::theMsg}()->\hyperlink{group__preregister_ga0162a39473e7f9b490a79a7983d949ac}{sendMsg}<getCallbackHandler>(to\_node, msg);
    \}

    \textcolor{comment}{// Example of active message handler callback with send node}
    \{
      \textcolor{keyword}{auto} cb = \hyperlink{namespacevt_a673b109e94c7bca58313504c83e1da94}{::vt::theCB}()->\hyperlink{structvt_1_1pipe_1_1_pipe_manager_a73583be6260418b13ee66e56cdade2da}{makeSend}<DataMsg,callbackHandler>(cb\_node);
      \textcolor{keyword}{auto} msg = ::vt::makeMessage<MsgWithCallback>(cb);
      \hyperlink{namespacevt_aeafd31f866aeb4dc6fc2f6ee97136350}{::vt::theMsg}()->\hyperlink{group__preregister_ga0162a39473e7f9b490a79a7983d949ac}{sendMsg}<getCallbackHandler>(to\_node, msg);
    \}

    \textcolor{comment}{// Example of active message handler callback with broadcast}
    \{
      \textcolor{keyword}{auto} cb = \hyperlink{namespacevt_a673b109e94c7bca58313504c83e1da94}{::vt::theCB}()->\hyperlink{structvt_1_1pipe_1_1_pipe_manager_a2ea6bd5ea3e001662681b1e5a1971e9d}{makeBcast}<DataMsg,callbackBcastHandler>();
      \textcolor{keyword}{auto} msg = ::vt::makeMessage<MsgWithCallback>(cb);
      \hyperlink{namespacevt_aeafd31f866aeb4dc6fc2f6ee97136350}{::vt::theMsg}()->\hyperlink{group__preregister_ga0162a39473e7f9b490a79a7983d949ac}{sendMsg}<getCallbackHandler>(to\_node, msg);
    \}

    \textcolor{comment}{// Example of context callback}
    \{
      \textcolor{keyword}{auto} cb = \hyperlink{namespacevt_a673b109e94c7bca58313504c83e1da94}{::vt::theCB}()->\hyperlink{structvt_1_1pipe_1_1_pipe_manager_a9ceec59c887d0fa1498b931c788962f6}{makeFunc}<DataMsg,MyContext>(
        \hyperlink{namespacevt_1_1pipe_acb42b284378c0fdac1d7c6335dc26f58ae1a9dc9f23534e63de9df0d540ac1611}{vt::pipe::LifetimeEnum::Once}, &\hyperlink{namespacevt_1_1config_a0551245b6b893932b95aaf8eac94eed1}{ctx}, callbackCtx
      );
      \textcolor{keyword}{auto} msg = ::vt::makeMessage<MsgWithCallback>(cb);
      \hyperlink{namespacevt_aeafd31f866aeb4dc6fc2f6ee97136350}{::vt::theMsg}()->\hyperlink{group__preregister_ga0162a39473e7f9b490a79a7983d949ac}{sendMsg}<getCallbackHandler>(to\_node, msg);
    \}
  \}
\}

\textcolor{comment}{// Message handler for to receive callback and invoke it}
\textcolor{keyword}{static} \textcolor{keywordtype}{void} getCallbackHandler(MsgWithCallback* msg) \{
  \textcolor{keyword}{auto} \textcolor{keyword}{const} cur\_node = \hyperlink{namespacevt_a26551fe0e6e6a1371111df5b12c7e92c}{::vt::theContext}()->\hyperlink{structvt_1_1ctx_1_1_context_a0d52c263ce8516546a67443d9a86fa5f}{getNode}();
  ::fmt::print(\textcolor{stringliteral}{"getCallbackHandler: triggered on node=\{\}\(\backslash\)n"}, cur\_node);

  \textcolor{comment}{// Send the callback a message}
  msg->cb.send();
\}
\end{DoxyCodeInclude}
\hypertarget{tutorial-1h}{}\subsection{Learning about reductions}\label{tutorial-1h}

\begin{DoxyCodeInclude}
\textcolor{comment}{//                       Reduce Message VT Base Class}
\textcolor{comment}{//                 \(\backslash\)--------------------------------------------/}
\textcolor{comment}{//                  \(\backslash\)                                          /}
\textcolor{comment}{//                   \(\backslash\)                            Reduce Data /}
\textcolor{comment}{//                    \(\backslash\)                          \(\backslash\)-----------/}
\textcolor{comment}{//                     \(\backslash\)                          \(\backslash\)         /}
\textcolor{keyword}{struct }ReduceDataMsg : \hyperlink{structvt_1_1collective_1_1reduce_1_1operators_1_1_reduce_t_msg}{::vt::collective::ReduceTMsg}<int32\_t> \{\};


\textcolor{comment}{// Functor that is the target of the reduction}
\textcolor{keyword}{struct }ReduceResult \{
  \textcolor{keywordtype}{void} operator()(ReduceDataMsg* msg) \{
    \hyperlink{namespacevt_a866da9d0efc19c0a1ce79e9e492f47e2}{NodeType} \textcolor{keyword}{const} num\_nodes = \hyperlink{namespacevt_a26551fe0e6e6a1371111df5b12c7e92c}{::vt::theContext}()->
      \hyperlink{structvt_1_1ctx_1_1_context_a7f41071aadf6d5fa9e1b6c703c5ff19d}{getNumNodes}();
    fmt::print(\textcolor{stringliteral}{"reduction value=\{\}\(\backslash\)n"}, msg->getConstVal());
    assert(num\_nodes * 50 == msg->getConstVal());
    (void)num\_nodes;  \textcolor{comment}{// don't warn about unused value when not debugging}
  \}
\};


\textcolor{comment}{// Tutorial code to demonstrate using reduction on all nodes}
\textcolor{keyword}{static} \textcolor{keyword}{inline} \textcolor{keywordtype}{void} activeMessageReduce() \{
  \hyperlink{namespacevt_a866da9d0efc19c0a1ce79e9e492f47e2}{NodeType} \textcolor{keyword}{const} this\_node = \hyperlink{namespacevt_a26551fe0e6e6a1371111df5b12c7e92c}{::vt::theContext}()->\hyperlink{structvt_1_1ctx_1_1_context_a0d52c263ce8516546a67443d9a86fa5f}{getNode}();
  (void)this\_node;  \textcolor{comment}{// don't warn about unused variable}
  \hyperlink{namespacevt_a866da9d0efc19c0a1ce79e9e492f47e2}{NodeType} \textcolor{keyword}{const} num\_nodes = \hyperlink{namespacevt_a26551fe0e6e6a1371111df5b12c7e92c}{::vt::theContext}()->
      \hyperlink{structvt_1_1ctx_1_1_context_a7f41071aadf6d5fa9e1b6c703c5ff19d}{getNumNodes}();
  (void)num\_nodes;  \textcolor{comment}{// don't warn about unused variable}

  \textcolor{comment}{/*}
\textcolor{comment}{   * Perform reduction over all the nodes.}
\textcolor{comment}{   */}

  \textcolor{comment}{// This is the type of the reduction (uses the plus operator over the data}
  \textcolor{comment}{// type). Once can implement their own data type and overload the plus}
  \textcolor{comment}{// operator for the combine during the reduce}
  \textcolor{keyword}{using} ReduceOp = \hyperlink{structvt_1_1collective_1_1reduce_1_1operators_1_1_plus_op}{::vt::collective::PlusOp<int32\_t>};

  \hyperlink{namespacevt_a866da9d0efc19c0a1ce79e9e492f47e2}{NodeType} \textcolor{keyword}{const} root\_reduce\_node = 0;

  \textcolor{keyword}{auto} reduce\_msg = ::vt::makeMessage<ReduceDataMsg>();

  \textcolor{comment}{// Get a reference to the value to set it in this reduce msg}
  reduce\_msg->getVal() = 50;

  \textcolor{keyword}{auto} \textcolor{keyword}{const} default\_proxy = \hyperlink{namespacevt_a833f0115b692f578167cbd88e30d39c5}{theObjGroup}()->\hyperlink{structvt_1_1objgroup_1_1_obj_group_manager_a4447b8f4881710bc7b0b7a37de9d10b9}{getDefault}();
  default\_proxy.\hyperlink{structvt_1_1objgroup_1_1proxy_1_1_proxy_3_01void_01_4_a545aeca8f6c0250221796b2138a4583d}{reduceMsg}<ReduceOp, ReduceResult>(
    root\_reduce\_node, reduce\_msg.get()
  );
\}
\end{DoxyCodeInclude}
\hypertarget{tutorial-2a}{}\subsection{Learning about collections}\label{tutorial-2a}

\begin{DoxyCodeInclude}
\textcolor{comment}{// Forward declaration for message}
\textcolor{keyword}{struct }MyCollMsg;

\textcolor{comment}{//               VT Base Class for a collection}
\textcolor{comment}{//         \(\backslash\)-------------------------------------/}
\textcolor{comment}{//          \(\backslash\)                                   /}
\textcolor{comment}{//           \(\backslash\)                         Index   /}
\textcolor{comment}{//            \(\backslash\)                     \(\backslash\)---------/}
\textcolor{comment}{//             \(\backslash\)                     \(\backslash\)       /}
\textcolor{keyword}{struct }MyCol : \hyperlink{structvt_1_1vrt_1_1collection_1_1_collection}{::vt::Collection}<MyCol,Index1D> \{

  \textcolor{keywordtype}{void} msgHandler(MyCollMsg* msg);

\};

\textcolor{comment}{//                 VT Base Message for Collections}
\textcolor{comment}{//               \(\backslash\)--------------------------------/}
\textcolor{comment}{//                \(\backslash\)                              /}
\textcolor{keyword}{struct }MyCollMsg : \hyperlink{structvt_1_1vrt_1_1collection_1_1_collection_message}{::vt::CollectionMessage}<MyCol> \{ \};

\textcolor{keywordtype}{void} MyCol::msgHandler(MyCollMsg* msg) \{
  \textcolor{keyword}{auto} cur\_node = \hyperlink{namespacevt_a26551fe0e6e6a1371111df5b12c7e92c}{theContext}()->\hyperlink{structvt_1_1ctx_1_1_context_a0d52c263ce8516546a67443d9a86fa5f}{getNode}();
  \textcolor{keyword}{auto} idx = this->\hyperlink{namespacecheckpoint_1_1dispatch_1_1typeregistry_aa75ef84e9b63a687784360ded2d56fe4}{getIndex}();
  ::fmt::print(\textcolor{stringliteral}{"MyCol::msgHandler index=\{\}, node=\{\}\(\backslash\)n"}, idx.x(), cur\_node);
\}

\textcolor{comment}{// Tutorial code to demonstrate creating a collection}
\textcolor{keyword}{static} \textcolor{keyword}{inline} \textcolor{keywordtype}{void} collection() \{
  \hyperlink{namespacevt_a866da9d0efc19c0a1ce79e9e492f47e2}{NodeType} \textcolor{keyword}{const} this\_node = \hyperlink{namespacevt_a26551fe0e6e6a1371111df5b12c7e92c}{::vt::theContext}()->\hyperlink{structvt_1_1ctx_1_1_context_a0d52c263ce8516546a67443d9a86fa5f}{getNode}();
  \hyperlink{namespacevt_a866da9d0efc19c0a1ce79e9e492f47e2}{NodeType} \textcolor{keyword}{const} num\_nodes = \hyperlink{namespacevt_a26551fe0e6e6a1371111df5b12c7e92c}{::vt::theContext}()->
      \hyperlink{structvt_1_1ctx_1_1_context_a7f41071aadf6d5fa9e1b6c703c5ff19d}{getNumNodes}();
  (void)num\_nodes;  \textcolor{comment}{// don't warn about unused variable}

  \textcolor{comment}{/*}
\textcolor{comment}{   * This is an example of creating a virtual context collection with an index}
\textcolor{comment}{   * range}
\textcolor{comment}{   */}

  \textcolor{comment}{// Range of 32 elements for the collection}
  \textcolor{keyword}{auto} range = \hyperlink{namespacevt_a5540efc78234273e1796fb003fe4d234}{vt::Index1D}(32);

  \textcolor{comment}{// Construct the collection (collective variant): invoked by all nodes. By}
  \textcolor{comment}{// default, the elements will be block mapped to the nodes}
  \textcolor{keyword}{auto} proxy = vt::makeCollection<MyCol>(\textcolor{stringliteral}{"tutorial\_collection"})
    .bounds(range)    \textcolor{comment}{// Set the bounds for the collection}
    .bulkInsert()     \textcolor{comment}{// Bulk insert all the elements within the bounds}
    .wait();          \textcolor{comment}{// Wait for construction and get the proxy back}

  \textcolor{keywordflow}{if} (this\_node == 0) \{
    \textcolor{comment}{// Broadcast a message to the entire collection. The msgHandler will be}
    \textcolor{comment}{// invoked on every element to the collection}
    proxy.\hyperlink{structvt_1_1objgroup_1_1proxy_1_1_proxy_a0b716ca776b1f06e0d7d45afbe9e5274}{broadcast}<MyCollMsg,&MyCol::msgHandler>();

    \textcolor{comment}{// Send a message to the 5th element of the collection}
    proxy[5].send<MyCollMsg,&MyCol::msgHandler>();
  \}

  \textcolor{comment}{// Collectively broadcast to the collection (all nodes participate for a}
  \textcolor{comment}{// single broadcast)}
  proxy.broadcastCollective<MyCollMsg,&MyCol::msgHandler>();
\}
\end{DoxyCodeInclude}
\hypertarget{tutorial-2b}{}\subsection{Learning about collection reductions}\label{tutorial-2b}

\begin{DoxyCodeInclude}
\textcolor{comment}{// Forward declaration for message}
\textcolor{keyword}{struct }ColRedMsg;

\textcolor{comment}{//                  VT Base Class for a collection}
\textcolor{comment}{//            \(\backslash\)------------------------------------------/}
\textcolor{comment}{//             \(\backslash\)                                        /}
\textcolor{comment}{//              \(\backslash\)                              Index   /}
\textcolor{comment}{//               \(\backslash\)                          \(\backslash\)---------/}
\textcolor{comment}{//                \(\backslash\)                          \(\backslash\)       /}
\textcolor{keyword}{struct }ReduceCol : \hyperlink{structvt_1_1vrt_1_1collection_1_1_collection}{::vt::Collection}<ReduceCol,Index1D> \{

  \textcolor{keywordtype}{void} reduceHandler(ColRedMsg* msg);

\};

\textcolor{comment}{//                  VT Base Message for Collections}
\textcolor{comment}{//               \(\backslash\)-----------------------------------/}
\textcolor{comment}{//                \(\backslash\)                                 /}
\textcolor{keyword}{struct }ColRedMsg : \hyperlink{structvt_1_1vrt_1_1collection_1_1_collection_message}{::vt::CollectionMessage}<ReduceCol> \{ \};

\textcolor{comment}{//                    Reduce Message VT Base Class}
\textcolor{comment}{//              \(\backslash\)-------------------------------------------/}
\textcolor{comment}{//               \(\backslash\)                                         /}
\textcolor{comment}{//                \(\backslash\)                           Reduce Type /}
\textcolor{comment}{//                 \(\backslash\)                          \(\backslash\)----------/}
\textcolor{comment}{//                  \(\backslash\)                          \(\backslash\)        /}
\textcolor{keyword}{struct }ReduceMsg : \hyperlink{structvt_1_1collective_1_1reduce_1_1operators_1_1_reduce_t_msg}{::vt::collective::ReduceTMsg}<int32\_t> \{\};

\textcolor{comment}{// Functor that is the target of the collection reduction}
\textcolor{keyword}{struct }PrintReduceResult \{
  \textcolor{keywordtype}{void} operator()(ReduceMsg* msg) \{
    fmt::print(\textcolor{stringliteral}{"collection reduce value=\{\}\(\backslash\)n"}, msg->getConstVal());
    assert(32 * 100 == msg->getConstVal());
  \}
\};


\textcolor{keywordtype}{void} ReduceCol::reduceHandler(ColRedMsg* msg) \{
  \textcolor{keyword}{auto} cur\_node = \hyperlink{namespacevt_a26551fe0e6e6a1371111df5b12c7e92c}{theContext}()->\hyperlink{structvt_1_1ctx_1_1_context_a0d52c263ce8516546a67443d9a86fa5f}{getNode}();
  (void)cur\_node;  \textcolor{comment}{// don't warn about unused variable}
  \textcolor{keyword}{auto} idx = this->\hyperlink{namespacecheckpoint_1_1dispatch_1_1typeregistry_aa75ef84e9b63a687784360ded2d56fe4}{getIndex}();
  (void)idx;  \textcolor{comment}{// don't warn about unused variable}

  \textcolor{comment}{//::fmt::print("MyCol::reduceHandler index=\{\}, node=\{\}\(\backslash\)n", idx.x(), cur\_node);}

  \textcolor{keyword}{using} ReduceOp = \hyperlink{structvt_1_1collective_1_1reduce_1_1operators_1_1_plus_op}{vt::collective::PlusOp<int32\_t>};

  \textcolor{keyword}{auto} proxy = getCollectionProxy();
  \textcolor{keyword}{auto} reduce\_msg = makeMessage<ReduceMsg>();

  \textcolor{comment}{// Get a reference to the value to set it in this reduce msg}
  reduce\_msg->getVal() = 100;

  \textcolor{comment}{// Invoke the reduce!}
  proxy.\hyperlink{structvt_1_1objgroup_1_1proxy_1_1_proxy_a61273d407174fb496b4aed10ec6650bd}{reduce}<ReduceOp,PrintReduceResult>(reduce\_msg.get());
\}

\textcolor{comment}{// Tutorial code to demonstrate reducing a collection}
\textcolor{keyword}{static} \textcolor{keyword}{inline} \textcolor{keywordtype}{void} collectionReduce() \{
  \hyperlink{namespacevt_a866da9d0efc19c0a1ce79e9e492f47e2}{NodeType} \textcolor{keyword}{const} this\_node = \hyperlink{namespacevt_a26551fe0e6e6a1371111df5b12c7e92c}{::vt::theContext}()->\hyperlink{structvt_1_1ctx_1_1_context_a0d52c263ce8516546a67443d9a86fa5f}{getNode}();
  \hyperlink{namespacevt_a866da9d0efc19c0a1ce79e9e492f47e2}{NodeType} \textcolor{keyword}{const} num\_nodes = \hyperlink{namespacevt_a26551fe0e6e6a1371111df5b12c7e92c}{::vt::theContext}()->
      \hyperlink{structvt_1_1ctx_1_1_context_a7f41071aadf6d5fa9e1b6c703c5ff19d}{getNumNodes}();
  (void)num\_nodes;  \textcolor{comment}{// don't warn about unused variable}

  \textcolor{comment}{/*}
\textcolor{comment}{   * This is an example of reducing over a virtual context collection}
\textcolor{comment}{   */}

  \textcolor{keywordflow}{if} (this\_node == 0) \{
    \textcolor{comment}{// Range of 32 elements for the collection}
    \textcolor{keyword}{auto} range = \hyperlink{namespacevt_a5540efc78234273e1796fb003fe4d234}{vt::Index1D}(32);

    \textcolor{comment}{// Construct the collection in a rooted manner. By default, the elements}
    \textcolor{comment}{// will be block mapped to the nodes}
    \textcolor{keyword}{auto} proxy = vt::makeCollectionRooted<ReduceCol>(\textcolor{stringliteral}{"tutorial\_collection\_reduce"})
      .bounds(range)     \textcolor{comment}{// Set the bounds for the collection}
      .bulkInsert()      \textcolor{comment}{// Bulk insert all the elements within the bounds}
      .wait();           \textcolor{comment}{// Wait for construction and get the proxy back}

    \textcolor{comment}{// Broadcast a message to the entire collection. The reduceHandler will be}
    \textcolor{comment}{// invoked on every element to the collection}
    proxy.\hyperlink{structvt_1_1objgroup_1_1proxy_1_1_proxy_a0b716ca776b1f06e0d7d45afbe9e5274}{broadcast}<ColRedMsg,&ReduceCol::reduceHandler>();
  \}
\}
\end{DoxyCodeInclude}
\hypertarget{tutorial-3a}{}\subsection{Learning about collective epochs}\label{tutorial-3a}

\begin{DoxyCodeInclude}
\textcolor{comment}{//              VT Base Message}
\textcolor{comment}{//             \(\backslash\)----------------/}
\textcolor{comment}{//              \(\backslash\)              /}
\textcolor{keyword}{struct }ExampleMsg : \hyperlink{structvt_1_1messaging_1_1_active_msg}{::vt::Message} \{
  ExampleMsg() = \textcolor{keywordflow}{default};
  \textcolor{keyword}{explicit} ExampleMsg(int32\_t in\_ttl) : ttl(in\_ttl-1) \{ \}

  int32\_t ttl = 0;
\};

\textcolor{comment}{// Forward declaration for the active message handler}
\textcolor{keyword}{static} \textcolor{keywordtype}{void} recurHandler(ExampleMsg* msg);

\textcolor{comment}{// Tutorial code to demonstrate using a callback}
\textcolor{keyword}{static} \textcolor{keyword}{inline} \textcolor{keywordtype}{void} activeMessageTerm() \{
  \hyperlink{namespacevt_a866da9d0efc19c0a1ce79e9e492f47e2}{NodeType} \textcolor{keyword}{const} this\_node = \hyperlink{namespacevt_a26551fe0e6e6a1371111df5b12c7e92c}{::vt::theContext}()->\hyperlink{structvt_1_1ctx_1_1_context_a0d52c263ce8516546a67443d9a86fa5f}{getNode}();
  \hyperlink{namespacevt_a866da9d0efc19c0a1ce79e9e492f47e2}{NodeType} \textcolor{keyword}{const} num\_nodes = \hyperlink{namespacevt_a26551fe0e6e6a1371111df5b12c7e92c}{::vt::theContext}()->
      \hyperlink{structvt_1_1ctx_1_1_context_a7f41071aadf6d5fa9e1b6c703c5ff19d}{getNumNodes}();
  (void)num\_nodes;  \textcolor{comment}{// don't warn about unused variable}

  \textcolor{comment}{/*}
\textcolor{comment}{   * Termination will allow us to track a subcomputation with causality to}
\textcolor{comment}{   * determine when a sub-computation terminated in a distributed-manner. The}
\textcolor{comment}{   * tutorial demonstrates how to use `collective` epochs. Rooted epoch will be}
\textcolor{comment}{   * demonstrated in a follow-on tutorial.}
\textcolor{comment}{   */}

  \textcolor{comment}{// Create a new epoch: this is a collective invocation}
  \textcolor{keyword}{auto} \textcolor{keyword}{const} new\_epoch = \hyperlink{namespacevt_a127580fdfcaba0b4171e5c48c5676733}{theTerm}()->\hyperlink{structvt_1_1term_1_1_termination_detector_aeba62165ec8dfbc445e3631ff295f159}{makeEpochCollective}();

  \textcolor{keywordflow}{if} (this\_node == 0) \{
    \textcolor{keyword}{auto} msg = vt::makeMessage<ExampleMsg>(8);
    \hyperlink{namespacevt_a4a0a9928690206b588dbcac2afb71088}{envelopeSetEpoch}(msg->env, new\_epoch);
    \hyperlink{namespacevt_aeafd31f866aeb4dc6fc2f6ee97136350}{vt::theMsg}()->\hyperlink{group__preregister_ga0162a39473e7f9b490a79a7983d949ac}{sendMsg}<recurHandler>(this\_node+1, msg);
  \}

  \textcolor{comment}{// Any node that wishes to have a notification on termination for a given}
  \textcolor{comment}{// epoch can add actions for the termination detector}
  \hyperlink{namespacevt_a127580fdfcaba0b4171e5c48c5676733}{theTerm}()->\hyperlink{structvt_1_1term_1_1_term_action_a1227042cb4eb38937fb8ed34bcbdf998}{addAction}(
    new\_epoch, []\{
      \textcolor{keyword}{auto} \textcolor{keyword}{const} \hyperlink{namespacevt_1_1config_a46e57c33bee1b9f1fa95455af86d80e0a576033c81e2ddd03d721ecef3b1b3f81}{node} = \hyperlink{namespacevt_a26551fe0e6e6a1371111df5b12c7e92c}{vt::theContext}()->\hyperlink{structvt_1_1ctx_1_1_context_a0d52c263ce8516546a67443d9a86fa5f}{getNode}();
      fmt::print(\textcolor{stringliteral}{"\{\}: recurHandler terminated\(\backslash\)n"}, \hyperlink{namespacevt_1_1config_a46e57c33bee1b9f1fa95455af86d80e0a576033c81e2ddd03d721ecef3b1b3f81}{node});
    \}
  );

  \textcolor{comment}{// This is not explicitly a collective, but all nodes need to call}
  \textcolor{comment}{// `finishedEpoch` to tell the system they are finished sending messages}
  \textcolor{comment}{// for the epoch.}
  \hyperlink{namespacevt_a127580fdfcaba0b4171e5c48c5676733}{theTerm}()->\hyperlink{structvt_1_1term_1_1_termination_detector_ad54d75c50bd3f34f30247817737bc303}{finishedEpoch}(new\_epoch);
\}

\textcolor{comment}{// Message handler that recursively sends messages}
\textcolor{keyword}{static} \textcolor{keywordtype}{void} recurHandler(ExampleMsg* msg) \{
  \hyperlink{namespacevt_a866da9d0efc19c0a1ce79e9e492f47e2}{NodeType} \textcolor{keyword}{const} num\_nodes = \hyperlink{namespacevt_a26551fe0e6e6a1371111df5b12c7e92c}{::vt::theContext}()->
      \hyperlink{structvt_1_1ctx_1_1_context_a7f41071aadf6d5fa9e1b6c703c5ff19d}{getNumNodes}();
  \hyperlink{namespacevt_a866da9d0efc19c0a1ce79e9e492f47e2}{NodeType} \textcolor{keyword}{const} this\_node = \hyperlink{namespacevt_a26551fe0e6e6a1371111df5b12c7e92c}{::vt::theContext}()->\hyperlink{structvt_1_1ctx_1_1_context_a0d52c263ce8516546a67443d9a86fa5f}{getNode}();

  ::fmt::print(
    \textcolor{stringliteral}{"\{\}: recurHandler: ttl=\{\}, triggered\(\backslash\)n"}, this\_node, msg->ttl
  );

  \textcolor{keywordflow}{if} (msg->ttl > 0) \{
    \textcolor{keyword}{auto} \textcolor{keyword}{const} num\_send = \textcolor{keyword}{static\_cast<}int32\_t\textcolor{keyword}{>}(drand48() * 3);
    \textcolor{keywordflow}{for} (\textcolor{keyword}{auto} i = 0; i < num\_send; i++) \{
      \textcolor{keyword}{auto} next\_node = (this\_node + 1 > num\_nodes - 1) ? 0 : (this\_node + 1);

      ::fmt::print(
        \textcolor{stringliteral}{"\{\}: recurHandler: i=\{\}, next\_node=\{\}, num\_send=\{\}\(\backslash\)n"},
        this\_node, i, next\_node, num\_send
      );

      \textcolor{keyword}{auto} msg\_send = vt::makeMessage<ExampleMsg>(msg->ttl);
      \hyperlink{namespacevt_aeafd31f866aeb4dc6fc2f6ee97136350}{vt::theMsg}()->\hyperlink{group__preregister_ga0162a39473e7f9b490a79a7983d949ac}{sendMsg}<recurHandler>(next\_node, msg\_send);
    \}
  \}
\}
\end{DoxyCodeInclude}
\hypertarget{examples}{}\section{Examples}\label{examples}
Small example programs to learn {\bfseries {\itshape vt}}

\tabulinesep=1mm
\begin{longtabu} spread 0pt [c]{*{3}{|X[-1]}|}
\hline
\rowcolor{\tableheadbgcolor}\textbf{ Example }&\textbf{ Description }&\textbf{ Link  }\\\cline{1-3}
\endfirsthead
\hline
\endfoot
\hline
\rowcolor{\tableheadbgcolor}\textbf{ Example }&\textbf{ Description }&\textbf{ Link  }\\\cline{1-3}
\endhead
Jacobi1D &1-\/D Jacobi linear system solver with homogeneous Dirichlet condition &\hyperlink{jacobi1d-example}{Example Jacobi 1D Program} \\\cline{1-3}
Jacobi2D &2-\/D Jacobi linear system solver with homogeneous Dirichlet condition &\hyperlink{jacobi2d-example}{Example Jacobi 2D Program} \\\cline{1-3}
Reduce\+Integral &Composite trapezoidal rule for the integral {\ttfamily \textbackslash{}int\+\_\+\{0\}$^\wedge$\{1\} f(x) dx} &\hyperlink{reduce-integral-example}{Example Integral Reduction Program} \\\cline{1-3}
\end{longtabu}
\hypertarget{jacobi1d-example}{}\subsection{Example Jacobi 1D Program}\label{jacobi1d-example}
The full code for this example can be found here\+: {\ttfamily examples/collection/jacobi1d\+\_\+vt.\+cc}.


\begin{DoxyCodeInclude}

\textcolor{comment}{//}
\textcolor{comment}{// This code applies a few steps of the Jacobi iteration to}
\textcolor{comment}{// the linear system  A x = 0}
\textcolor{comment}{// where is the tridiagonal matrix with pattern [-1 2 -1]}
\textcolor{comment}{// The initial guess for x is a made-up non-zero vector.}
\textcolor{comment}{// The exact solution is the vector 0.}
\textcolor{comment}{//}
\textcolor{comment}{// The matrix A is square and invertible.}
\textcolor{comment}{// The number of rows is ((number of objects) * (number of rows per object))}
\textcolor{comment}{//}
\textcolor{comment}{// Such a matrix A is obtained when using 2nd-order finite difference}
\textcolor{comment}{// for discretizing (-d^2 u /dx^2 = f) on [0, 1] with homogeneous}
\textcolor{comment}{// Dirichlet condition (u(0) = u(1) = 0) using a uniform grid}
\textcolor{comment}{// with grid size 1 / ((number of objects) * (number of rows per object) + 1)}
\textcolor{comment}{//}


\textcolor{preprocessor}{#include <\hyperlink{transport_8h}{vt/transport.h}>}
\textcolor{preprocessor}{#include <\hyperlink{invoke_8h}{vt/runnable/invoke.h}>}

\textcolor{preprocessor}{#include <cstdlib>}
\textcolor{preprocessor}{#include <cassert>}
\textcolor{preprocessor}{#include <iostream>}

\textcolor{keyword}{static} constexpr std::size\_t \textcolor{keyword}{const} default\_nrow\_object = 8;
\textcolor{keyword}{static} constexpr std::size\_t \textcolor{keyword}{const} default\_num\_objs = 4;
\textcolor{keyword}{static} constexpr \textcolor{keywordtype}{double} \textcolor{keyword}{const} default\_tol = 1.0e-02;

\textcolor{keyword}{struct }NodeObj \{
  \textcolor{keywordtype}{bool} is\_finished\_ = \textcolor{keyword}{false};
  \textcolor{keyword}{struct }WorkFinishedMsg : \hyperlink{structvt_1_1messaging_1_1_active_msg}{vt::Message} \{\};

  \textcolor{keywordtype}{void} workFinishedHandler(WorkFinishedMsg*) \{ is\_finished\_ = \textcolor{keyword}{true}; \}
  \textcolor{keywordtype}{bool} isWorkFinished() \{ \textcolor{keywordflow}{return} is\_finished\_; \}
\};
\textcolor{keyword}{using} NodeObjProxy = \hyperlink{structvt_1_1objgroup_1_1proxy_1_1_proxy}{vt::objgroup::proxy::Proxy<NodeObj>};

\textcolor{keyword}{struct }LinearPb1DJacobi : \hyperlink{structvt_1_1vrt_1_1collection_1_1_collection}{vt::Collection}<LinearPb1DJacobi,vt::Index1D> \{

\textcolor{keyword}{private}:

  std::vector<double> tcur\_, told\_;
  std::vector<double> rhs\_;
  \textcolor{keywordtype}{size\_t} iter\_ = 0;
  \textcolor{keywordtype}{size\_t} msgReceived\_ = 0, totalReceive\_ = 0;
  \textcolor{keywordtype}{size\_t} numObjs\_ = 1;
  \textcolor{keywordtype}{size\_t} numRowsPerObject\_ = 1;
  \textcolor{keywordtype}{size\_t} maxIter\_ = 8;
  NodeObjProxy objProxy\_;

\textcolor{keyword}{public}:

  \textcolor{keyword}{explicit} LinearPb1DJacobi() :
    tcur\_(), told\_(), rhs\_(), iter\_(0),
    msgReceived\_(0), totalReceive\_(0),
    numObjs\_(1), numRowsPerObject\_(1), maxIter\_(8)
  \{ \}


  \textcolor{keyword}{using} BlankMsg = \hyperlink{structvt_1_1vrt_1_1collection_1_1_collection_message}{vt::CollectionMessage<LinearPb1DJacobi>};

  \textcolor{keyword}{struct }LPMsg : \hyperlink{structvt_1_1vrt_1_1collection_1_1_collection_message}{vt::CollectionMessage}<LinearPb1DJacobi> \{

    \textcolor{keywordtype}{size\_t} numObjects = 0;
    \textcolor{keywordtype}{size\_t} nRowPerObject = 0;
    \textcolor{keywordtype}{size\_t} iterMax = 0;
    NodeObjProxy objProxy;

    LPMsg() = \textcolor{keywordflow}{default};

    LPMsg(\textcolor{keyword}{const} \textcolor{keywordtype}{size\_t} nobjs, \textcolor{keyword}{const} \textcolor{keywordtype}{size\_t} nrow, \textcolor{keyword}{const} \textcolor{keywordtype}{size\_t} itMax, NodeObjProxy proxy) :
      numObjects(nobjs), nRowPerObject(nrow), iterMax(itMax), objProxy(proxy)
    \{ \}

  \};

  \textcolor{keyword}{struct }ReduxMsg : \hyperlink{structvt_1_1collective_1_1reduce_1_1operators_1_1_reduce_t_msg}{vt::collective::ReduceTMsg}<double> \{
    ReduxMsg() = \textcolor{keywordflow}{default};
    \textcolor{keyword}{explicit} ReduxMsg(\textcolor{keywordtype}{double} in\_val) : \hyperlink{namespacevt_1_1collective_a28b82d5d48c9bc6e4fd738fcbf9e0f62}{ReduceTMsg}<double>(in\_val) \{ \}
  \};

  \textcolor{keywordtype}{void} checkCompleteCB(ReduxMsg* msg) \{
    \textcolor{comment}{//}
    \textcolor{comment}{// Only one object for the reduction will visit}
    \textcolor{comment}{// this function}
    \textcolor{comment}{//}

    \textcolor{keywordtype}{double} normRes = msg->getConstVal();

    \textcolor{keyword}{auto} \textcolor{keyword}{const} iter\_max\_reached = iter\_ > maxIter\_;
    \textcolor{keyword}{auto} \textcolor{keyword}{const} norm\_res\_done = normRes < default\_tol;

    \textcolor{keywordflow}{if} (iter\_max\_reached or norm\_res\_done) \{
      \textcolor{keyword}{auto} \textcolor{keyword}{const} to\_print = iter\_max\_reached ?
        \textcolor{stringliteral}{"\(\backslash\)n Maximum Number of Iterations Reached. \(\backslash\)n\(\backslash\)n"} :
        fmt::format(\textcolor{stringliteral}{"\(\backslash\)n Max-Norm Residual Reduced by \{\} \(\backslash\)n\(\backslash\)n"}, default\_tol);

      fmt::print(to\_print);

      \textcolor{comment}{// Notify all nodes that computation is finished}
      objProxy\_.broadcast<NodeObj::WorkFinishedMsg, &NodeObj::workFinishedHandler>();
    \} \textcolor{keywordflow}{else} \{
      fmt::print(\textcolor{stringliteral}{" ## ITER \{\} >> Residual Norm = \{\} \(\backslash\)n"}, iter\_, normRes);
    \}
  \}

  \textcolor{keywordtype}{void} doIteration() \{

    iter\_ += 1;

    \textcolor{comment}{//}
    \textcolor{comment}{//--- Copy extremal values}
    \textcolor{comment}{//}
    tcur\_[0] = told\_[0];
    tcur\_[numRowsPerObject\_+1] = told\_[numRowsPerObject\_+1];

    \textcolor{comment}{//}
    \textcolor{comment}{//---- Jacobi iteration step}
    \textcolor{comment}{//---- A tridiagonal matrix = "tridiag" ( [-1.0  2.0  -1.0] )}
    \textcolor{comment}{//---- rhs\_ right hand side vector}
    \textcolor{comment}{//}
    \textcolor{keywordflow}{for} (\textcolor{keywordtype}{size\_t} ii = 1; ii <= numRowsPerObject\_; ++ii) \{
      tcur\_[ii] = 0.5*(rhs\_[ii] + told\_[ii-1] + told\_[ii+1]);
    \}

    std::copy(tcur\_.begin(), tcur\_.end(), told\_.begin());

    \textcolor{comment}{//}
    \textcolor{comment}{// Compute the maximum entries among the rows on this object}
    \textcolor{comment}{// We do not take into account the "ghost" entries}
    \textcolor{comment}{// as they may be "out of date".}
    \textcolor{comment}{//}

    \textcolor{keywordtype}{double} maxNorm = 0.0;

    \textcolor{keywordflow}{for} (\textcolor{keywordtype}{size\_t} ii = 1; ii < tcur\_.size()-1; ++ii) \{
      \textcolor{keywordtype}{double} val = tcur\_[ii];
      maxNorm = (maxNorm > std::fabs(val)) ? maxNorm : std::fabs(val);
    \}

    \textcolor{keyword}{auto} proxy = this->getCollectionProxy();
    \textcolor{keyword}{auto} cb = \hyperlink{namespacevt_a673b109e94c7bca58313504c83e1da94}{vt::theCB}()->\hyperlink{structvt_1_1pipe_1_1_pipe_manager_a73583be6260418b13ee66e56cdade2da}{makeSend}<
      LinearPb1DJacobi,ReduxMsg,&LinearPb1DJacobi::checkCompleteCB
    >(proxy[0]);
    \textcolor{keyword}{auto} msg2 = vt::makeMessage<ReduxMsg>(maxNorm);
    proxy.\hyperlink{structvt_1_1objgroup_1_1proxy_1_1_proxy_a61273d407174fb496b4aed10ec6650bd}{reduce}<\hyperlink{structvt_1_1collective_1_1reduce_1_1operators_1_1_max_op}{vt::collective::MaxOp<double>}>(msg2.get(),cb);

  \}


  \textcolor{keyword}{struct }VecMsg : \hyperlink{structvt_1_1vrt_1_1collection_1_1_collection_message}{vt::CollectionMessage}<LinearPb1DJacobi> \{
    \textcolor{keyword}{using} MessageParentType = \hyperlink{structvt_1_1vrt_1_1collection_1_1_collection_message}{vt::CollectionMessage<LinearPb1DJacobi>}
      ;
    \hyperlink{message__serialize_8h_a3452798707ea3a45f9da58f7b4673990}{vt\_msg\_serialize\_if\_needed\_by\_parent\_or\_type1}(
      \hyperlink{namespacevt_afb96657e28fa98eb685c5e0c6b1b122e}{vt::IdxBase});

    VecMsg() = \textcolor{keywordflow}{default};

    VecMsg(\hyperlink{namespacevt_afb96657e28fa98eb685c5e0c6b1b122e}{vt::IdxBase} \textcolor{keyword}{const}& in\_index, \textcolor{keywordtype}{double} \textcolor{keyword}{const}& ref) :
      \hyperlink{namespacevt}{vt}::\hyperlink{namespacevt_ae7700e12f79c0fec16964aab84838428}{CollectionMessage}<LinearPb1DJacobi>(),
      from\_index(in\_index), val(ref)
    \{ \}

    \textcolor{keyword}{template} <\textcolor{keyword}{typename} Serializer>
    \textcolor{keywordtype}{void} \hyperlink{namespacecheckpoint_a075da4e7344cf037943362517e606c3a}{serialize}(Serializer& s) \{
      \hyperlink{namespacecheckpoint_a075da4e7344cf037943362517e606c3a}{MessageParentType::serialize}(s);
      s | from\_index;
      s | val;
    \}

    \hyperlink{namespacevt_afb96657e28fa98eb685c5e0c6b1b122e}{vt::IdxBase} from\_index = 0;
    \textcolor{keywordtype}{double} val = 0.0;
  \};

  \textcolor{keywordtype}{void} exchange(VecMsg *msg) \{
    \textcolor{comment}{// Receive and treat the message from a neighboring object.}

    \textcolor{keyword}{const} \hyperlink{namespacevt_afb96657e28fa98eb685c5e0c6b1b122e}{vt::IdxBase} myIdx = \hyperlink{namespacecheckpoint_1_1dispatch_1_1typeregistry_aa75ef84e9b63a687784360ded2d56fe4}{getIndex}().x();

    \textcolor{keywordflow}{if} (myIdx > msg->from\_index) \{
      this->told\_[0] = msg->val;
      msgReceived\_ += 1;
    \}

    \textcolor{keywordflow}{if} (myIdx < msg->from\_index) \{
      this->told\_[numRowsPerObject\_ + 1] = msg->val;
      msgReceived\_ += 1;
    \}

    \textcolor{comment}{// Check whether this 'object' has received all the expected messages.}
    \textcolor{keywordflow}{if} (msgReceived\_ == totalReceive\_) \{
      msgReceived\_ = 0;
      doIteration();
    \}

  \}

  \textcolor{keywordtype}{void} doIter(BlankMsg *msg) \{

    \textcolor{comment}{//}
    \textcolor{comment}{// Treat the particular case of 1 object}
    \textcolor{comment}{// where no communication is needed.}
    \textcolor{comment}{// Without this treatment, the code would not iterate.}
    \textcolor{comment}{//}

    \textcolor{keywordflow}{if} (numObjs\_ == 1) \{
      doIteration();
      \textcolor{keywordflow}{return};
    \}
    \textcolor{comment}{//---------------------------------------}

    \textcolor{comment}{//}
    \textcolor{comment}{// Routine to send information to a different object}
    \textcolor{comment}{//}

    \hyperlink{namespacevt_afb96657e28fa98eb685c5e0c6b1b122e}{vt::IdxBase} \textcolor{keyword}{const} myIdx = \hyperlink{namespacecheckpoint_1_1dispatch_1_1typeregistry_aa75ef84e9b63a687784360ded2d56fe4}{getIndex}().x();

    \textcolor{comment}{//--- Send the values to the left}
    \textcolor{keyword}{auto} proxy = this->getCollectionProxy();
    \textcolor{keywordflow}{if} (myIdx > 0) \{
      proxy[myIdx - 1].send<VecMsg, &LinearPb1DJacobi::exchange>(
        myIdx, told\_[1]
      );
    \}

    \textcolor{comment}{//--- Send values to the right}
    \textcolor{keywordflow}{if} (\textcolor{keywordtype}{size\_t}(myIdx) < numObjs\_ - 1) \{
      proxy[myIdx + 1].send<VecMsg, &LinearPb1DJacobi::exchange>(
        myIdx, told\_[numRowsPerObject\_]
      );
    \}
  \}


  \textcolor{keywordtype}{void} init() \{

    tcur\_.assign(numRowsPerObject\_ + 2, 0.0);
    told\_.assign(numRowsPerObject\_ + 2, 0.0);
    rhs\_.assign(numRowsPerObject\_ + 2, 0.0);

    \textcolor{keywordtype}{double} h = 1.0 / (numRowsPerObject\_ * numObjs\_ + 1.0);
    \textcolor{keywordtype}{int} nf = 3 * int(numRowsPerObject\_ * numObjs\_ + 1) / 4;

    \textcolor{keywordtype}{size\_t} \textcolor{keyword}{const} myIdx = \hyperlink{namespacecheckpoint_1_1dispatch_1_1typeregistry_aa75ef84e9b63a687784360ded2d56fe4}{getIndex}().x();

    \textcolor{keywordflow}{for} (\textcolor{keywordtype}{size\_t} ii = 0; ii < tcur\_.size(); ++ii) \{
      \textcolor{keywordtype}{double} x0 = ( numRowsPerObject\_ * myIdx + ii) * h;
      tcur\_[ii] = sin(nf * M\_PI * x0 * x0);
    \}

    totalReceive\_ = 2;

    \textcolor{keywordflow}{if} (myIdx == 0) \{
      tcur\_[0] = 0.0;
      totalReceive\_ -= 1;
    \}

    \textcolor{keywordflow}{if} (myIdx == numObjs\_ - 1) \{
      tcur\_[numRowsPerObject\_+1] = 0.0;
      totalReceive\_ -= 1;
    \}

    std::copy(tcur\_.begin(), tcur\_.end(), told\_.begin());

  \}


  \textcolor{keywordtype}{void} init(LPMsg* msg) \{
    numObjs\_ = msg->numObjects;
    numRowsPerObject\_ = msg->nRowPerObject;
    maxIter\_ = msg->iterMax;
    objProxy\_ = msg->objProxy;

    \textcolor{comment}{// Initialize the starting vector}
    init();
  \}

\};

\textcolor{keywordtype}{bool} isWorkDone( \hyperlink{structvt_1_1objgroup_1_1proxy_1_1_proxy}{vt::objgroup::proxy::Proxy<NodeObj>} \textcolor{keyword}{const}& proxy)\{
  \textcolor{keyword}{auto} \textcolor{keyword}{const} this\_node = \hyperlink{namespacevt_a26551fe0e6e6a1371111df5b12c7e92c}{vt::theContext}()->\hyperlink{structvt_1_1ctx_1_1_context_a0d52c263ce8516546a67443d9a86fa5f}{getNode}();
  \textcolor{keywordflow}{return} proxy[this\_node].invoke<&NodeObj::isWorkFinished>();
\}

\textcolor{keywordtype}{int} main(\textcolor{keywordtype}{int} argc, \textcolor{keywordtype}{char}** argv) \{

  \textcolor{keywordtype}{size\_t} num\_objs = default\_num\_objs;
  \textcolor{keywordtype}{size\_t} numRowsPerObject = default\_nrow\_object;
  \textcolor{keywordtype}{size\_t} maxIter = 8;

  std::string name(argv[0]);

  \hyperlink{namespacevt_a580ef23e9a224fc0907da1a6db308b5b}{vt::initialize}(argc, argv);

  \hyperlink{namespacevt_a866da9d0efc19c0a1ce79e9e492f47e2}{vt::NodeType} this\_node = \hyperlink{namespacevt_a26551fe0e6e6a1371111df5b12c7e92c}{vt::theContext}()->\hyperlink{structvt_1_1ctx_1_1_context_a0d52c263ce8516546a67443d9a86fa5f}{getNode}();
  \hyperlink{namespacevt_a866da9d0efc19c0a1ce79e9e492f47e2}{vt::NodeType} num\_nodes = \hyperlink{namespacevt_a26551fe0e6e6a1371111df5b12c7e92c}{vt::theContext}()->
      \hyperlink{structvt_1_1ctx_1_1_context_a7f41071aadf6d5fa9e1b6c703c5ff19d}{getNumNodes}();

  \textcolor{keywordflow}{if} (argc == 1) \{
    \textcolor{keywordflow}{if} (this\_node == 0) \{
      fmt::print(
        stderr, \textcolor{stringliteral}{"\{\}: using default arguments since none provided\(\backslash\)n"}, name
      );
    \}
    num\_objs = default\_num\_objs * num\_nodes;
  \} \textcolor{keywordflow}{else} \textcolor{keywordflow}{if} (argc == 2) \{
    num\_objs = \textcolor{keyword}{static\_cast<}\textcolor{keywordtype}{size\_t}\textcolor{keyword}{>}(strtol(argv[1], \textcolor{keyword}{nullptr}, 10));
  \}
  \textcolor{keywordflow}{else} \textcolor{keywordflow}{if} (argc == 3) \{
    num\_objs = \textcolor{keyword}{static\_cast<}\textcolor{keywordtype}{size\_t}\textcolor{keyword}{>}(strtol(argv[1], \textcolor{keyword}{nullptr}, 10));
    numRowsPerObject = \textcolor{keyword}{static\_cast<}\textcolor{keywordtype}{size\_t}\textcolor{keyword}{>}(strtol(argv[2], \textcolor{keyword}{nullptr}, 10));
  \}
  \textcolor{keywordflow}{else} \textcolor{keywordflow}{if} (argc == 4) \{
    num\_objs = \textcolor{keyword}{static\_cast<}\textcolor{keywordtype}{size\_t}\textcolor{keyword}{>}(strtol(argv[1], \textcolor{keyword}{nullptr}, 10));
    numRowsPerObject = \textcolor{keyword}{static\_cast<}\textcolor{keywordtype}{size\_t}\textcolor{keyword}{>}(strtol(argv[2], \textcolor{keyword}{nullptr}, 10));
    maxIter = \textcolor{keyword}{static\_cast<}\textcolor{keywordtype}{size\_t}\textcolor{keyword}{>}(strtol(argv[3], \textcolor{keyword}{nullptr}, 10));
  \}
  \textcolor{keywordflow}{else} \{
    fmt::print(
      stderr, \textcolor{stringliteral}{"usage: \{\} <num-objects> <num-rows-per-object> <maxiter>\(\backslash\)n"},
      name
    );
    \textcolor{keywordflow}{return} 1;
  \}

  \textcolor{comment}{// Object group of all nodes that take part in computation}
  \textcolor{comment}{// Used to determine whether the computation is finished}
  \textcolor{keyword}{auto} grp\_proxy = \hyperlink{namespacevt_a833f0115b692f578167cbd88e30d39c5}{vt::theObjGroup}()->\hyperlink{structvt_1_1objgroup_1_1_obj_group_manager_a50e0d8498d142a05773ed5a046803b2b}{makeCollective}<NodeObj>(\textcolor{stringliteral}{"
      examples\_jacobi1d"});

  \textcolor{comment}{// Create the decomposition into objects}
  \textcolor{keyword}{using} BaseIndexType = \textcolor{keyword}{typename} \hyperlink{structvt_1_1index_1_1_dense_index_array_a36698427e28045290d1fb072573275ec}{vt::Index1D::DenseIndexType};
  \textcolor{keyword}{auto} range = \hyperlink{namespacevt_a5540efc78234273e1796fb003fe4d234}{vt::Index1D}(static\_cast<BaseIndexType>(num\_objs));

  \textcolor{keyword}{auto} col\_proxy = vt::makeCollection<LinearPb1DJacobi>(\textcolor{stringliteral}{"examples\_jacobi1d"})
    .bounds(range)
    .bulkInsert()
    .wait();

  \hyperlink{namespacevt_a2fc4ef34f30b49a1781d765804bfadbb}{vt::runInEpochCollective}([col\_proxy, grp\_proxy, num\_objs, numRowsPerObject, 
      maxIter]\{
    col\_proxy.broadcastCollective<LinearPb1DJacobi::LPMsg, &LinearPb1DJacobi::init>(
      num\_objs, numRowsPerObject, maxIter, grp\_proxy
    );
  \});

  \textcolor{keywordflow}{while}(!isWorkDone(grp\_proxy))\{
    \hyperlink{namespacevt_a2fc4ef34f30b49a1781d765804bfadbb}{vt::runInEpochCollective}([col\_proxy]\{
      col\_proxy.broadcastCollective<
        LinearPb1DJacobi::BlankMsg, &LinearPb1DJacobi::doIter
      >();
    \});

    \hyperlink{namespacevt_a3799d6b51fb71d7b6505760aad457e61}{vt::thePhase}()->\hyperlink{structvt_1_1phase_1_1_phase_manager_ac94bfde4a103cbdec426542191c41092}{nextPhaseCollective}();
  \}

  \hyperlink{namespacevt_a540d90dbd6e97b69f1dcbc9ee9314cff}{vt::finalize}();

  \textcolor{keywordflow}{return} 0;
\}

\end{DoxyCodeInclude}
\hypertarget{jacobi2d-example}{}\subsection{Example Jacobi 2D Program}\label{jacobi2d-example}
The full code for this example can be found here\+: {\ttfamily examples/collection/jacobi2d\+\_\+vt.\+cc}.


\begin{DoxyCodeInclude}

\textcolor{comment}{//}
\textcolor{comment}{// This code applies a few steps of the Jacobi iteration to}
\textcolor{comment}{// the linear system  A x = 0}
\textcolor{comment}{// where is a banded symmetric positive definite matrix.}
\textcolor{comment}{// The initial guess for x is a made-up non-zero vector.}
\textcolor{comment}{// The exact solution is the vector 0.}
\textcolor{comment}{//}
\textcolor{comment}{// The matrix A is square and invertible.}
\textcolor{comment}{// The number of rows is ((number of objects) * (number of rows per object))}
\textcolor{comment}{//}
\textcolor{comment}{// Such a matrix A is obtained when using 2nd-order finite difference}
\textcolor{comment}{// for discretizing}
\textcolor{comment}{//}
\textcolor{comment}{// -d^2 u / dx^2 -d^2 u / dy^2 = f   on  [0, 1] x [0, 1]}
\textcolor{comment}{//}
\textcolor{comment}{// with homogeneous Dirichlet condition}
\textcolor{comment}{//}
\textcolor{comment}{// u = 0 on the boundary of [0, 1] x [0, 1]}
\textcolor{comment}{//}
\textcolor{comment}{// using a uniform grid with grid size}
\textcolor{comment}{//}
\textcolor{comment}{// 1 / ((number of objects) * (number of rows per object) + 1)}
\textcolor{comment}{//}


\textcolor{keyword}{static} constexpr std::size\_t \textcolor{keyword}{const} default\_nrow\_object = 8;
\textcolor{keyword}{static} constexpr std::size\_t \textcolor{keyword}{const} default\_num\_objs = 4;
\textcolor{keyword}{static} constexpr \textcolor{keywordtype}{double} \textcolor{keyword}{const} default\_tol = 1.0e-02;

\textcolor{keyword}{struct }NodeObj \{
  \textcolor{keywordtype}{bool} is\_finished\_ = \textcolor{keyword}{false};
  \textcolor{keyword}{struct }WorkFinishedMsg : \hyperlink{structvt_1_1messaging_1_1_active_msg}{vt::Message} \{\};

  \textcolor{keywordtype}{void} workFinishedHandler(WorkFinishedMsg*) \{ is\_finished\_ = \textcolor{keyword}{true}; \}
  \textcolor{keywordtype}{bool} isWorkFinished() \{ \textcolor{keywordflow}{return} is\_finished\_; \}
\};
\textcolor{keyword}{using} NodeObjProxy = \hyperlink{structvt_1_1objgroup_1_1proxy_1_1_proxy}{vt::objgroup::proxy::Proxy<NodeObj>};

\textcolor{keyword}{struct }LinearPb2DJacobi : \hyperlink{structvt_1_1vrt_1_1collection_1_1_collection}{vt::Collection}<LinearPb2DJacobi,vt::Index2D> \{

\textcolor{keyword}{private}:

  std::vector<double> tcur\_, told\_;
  std::vector<double> rhs\_;
  \textcolor{keywordtype}{size\_t} iter\_ = 0;
  \textcolor{keywordtype}{size\_t} msgReceived\_ = 0, totalReceive\_ = 0;
  \textcolor{keywordtype}{size\_t} numObjsX\_ = 1, numObjsY\_ = 1;
  \textcolor{keywordtype}{size\_t} numRowsPerObject\_ = default\_nrow\_object;
  \textcolor{keywordtype}{size\_t} maxIter\_ = 5;
  NodeObjProxy objProxy\_;

\textcolor{keyword}{public}:

  LinearPb2DJacobi()
    : tcur\_(), told\_(), rhs\_(), iter\_(0),
      msgReceived\_(0), totalReceive\_(0),
      numObjsX\_(1), numObjsY\_(1),
      numRowsPerObject\_(default\_nrow\_object),
      maxIter\_(5)
  \{ \}


  \textcolor{keyword}{struct }BlankMsg : \hyperlink{structvt_1_1vrt_1_1collection_1_1_collection_message}{vt::CollectionMessage}<LinearPb2DJacobi> \{ \};

  \textcolor{keyword}{struct }LPMsg : \hyperlink{structvt_1_1vrt_1_1collection_1_1_collection_message}{vt::CollectionMessage}<LinearPb2DJacobi> \{

    \textcolor{keywordtype}{size\_t} numXObjs = 0;
    \textcolor{keywordtype}{size\_t} numYObjs = 0;
    \textcolor{keywordtype}{size\_t} numIter = 0;
    NodeObjProxy objProxy;

    LPMsg() = \textcolor{keywordflow}{default};

    LPMsg(\textcolor{keyword}{const} \textcolor{keywordtype}{size\_t} nx, \textcolor{keyword}{const} \textcolor{keywordtype}{size\_t} ny, \textcolor{keyword}{const} \textcolor{keywordtype}{size\_t} nref, NodeObjProxy proxy)
      : numXObjs(nx), numYObjs(ny), numIter(nref), objProxy(proxy)
    \{ \}

  \};


  \textcolor{keyword}{struct }ReduxMsg : \hyperlink{structvt_1_1collective_1_1reduce_1_1operators_1_1_reduce_t_msg}{vt::collective::ReduceTMsg}<double> \{
    ReduxMsg() = \textcolor{keywordflow}{default};
    \textcolor{keyword}{explicit} ReduxMsg(\textcolor{keywordtype}{double} in\_val) : \hyperlink{namespacevt_1_1collective_a28b82d5d48c9bc6e4fd738fcbf9e0f62}{ReduceTMsg}<double>(in\_val) \{ \}
  \};


  \textcolor{keywordtype}{void} checkCompleteCB(ReduxMsg* msg) \{
    \textcolor{comment}{//}
    \textcolor{comment}{// Only one object for the reduction will visit}
    \textcolor{comment}{// this function}
    \textcolor{comment}{//}

    \textcolor{keyword}{const} \textcolor{keywordtype}{double} normRes = msg->getConstVal();
    \textcolor{keyword}{auto} \textcolor{keyword}{const} iter\_max\_reached = iter\_ > maxIter\_;
    \textcolor{keyword}{auto} \textcolor{keyword}{const} norm\_res\_done = normRes < default\_tol;

    \textcolor{keywordflow}{if} (iter\_max\_reached or norm\_res\_done) \{
      \textcolor{keyword}{auto} \textcolor{keyword}{const} to\_print = iter\_max\_reached ?
        \textcolor{stringliteral}{"\(\backslash\)n Maximum Number of Iterations Reached. \(\backslash\)n\(\backslash\)n"} :
        fmt::format(\textcolor{stringliteral}{"\(\backslash\)n Max-Norm Residual Reduced by \{\} \(\backslash\)n\(\backslash\)n"}, default\_tol);

      fmt::print(to\_print);

      \textcolor{comment}{// Notify all nodes that computation is finished}
      objProxy\_.broadcast<NodeObj::WorkFinishedMsg, &NodeObj::workFinishedHandler>();
    \} \textcolor{keywordflow}{else} \{
      fmt::print(\textcolor{stringliteral}{" ## ITER \{\} >> Residual Norm = \{\} \(\backslash\)n"}, iter\_, normRes);
    \}
  \}

  \textcolor{keywordtype}{void} doIteration() \{

    \textcolor{comment}{//}
    \textcolor{comment}{//--- Copy ghost values}
    \textcolor{comment}{//}

    \textcolor{keywordtype}{size\_t} ldx = numRowsPerObject\_ + 2;
    \textcolor{keywordtype}{size\_t} ldy = numRowsPerObject\_ + 2;

    \textcolor{keywordflow}{for} (\textcolor{keywordtype}{size\_t} jx = 0; jx < ldx; ++jx)
      tcur\_[jx] = told\_[jx];

    \textcolor{keywordflow}{for} (\textcolor{keywordtype}{size\_t} jx = 0; jx < ldx; ++jx)
      tcur\_[jx + (ldy-1) * ldx] = told\_[jx + (ldy-1) * ldx];

    \textcolor{keywordflow}{for} (\textcolor{keywordtype}{size\_t} jy = 0; jy < ldy; ++jy)
      tcur\_[jy * ldx] = told\_[jy * ldx];

    \textcolor{keywordflow}{for} (\textcolor{keywordtype}{size\_t} jy = 0; jy < ldy; ++jy)
      tcur\_[ldx-1 + jy * ldx] = told\_[ldx-1 + jy * ldx];

    \textcolor{comment}{//}
    \textcolor{comment}{//--- Update my row values}
    \textcolor{comment}{//}

    \textcolor{keywordflow}{for} (\textcolor{keywordtype}{size\_t} iy = 1; iy <= numRowsPerObject\_; ++iy) \{
      \textcolor{keywordflow}{for} (\textcolor{keywordtype}{size\_t} ix = 1; ix <= numRowsPerObject\_; ++ix) \{
        \textcolor{comment}{//}
        \textcolor{comment}{//---- Jacobi iteration step for}
        \textcolor{comment}{//---- A banded matrix for the 5-point stencil}
        \textcolor{comment}{//---- [ 0.0  -1.0   0.0]}
        \textcolor{comment}{//---- [-1.0   4.0  -1.0]}
        \textcolor{comment}{//---- [ 0.0  -1.0   0.0]}
        \textcolor{comment}{//---- rhs\_ right hand side vector}
        \textcolor{comment}{//}
        \textcolor{keywordtype}{size\_t} \hyperlink{namespacevt_1_1config_a46e57c33bee1b9f1fa95455af86d80e0a576033c81e2ddd03d721ecef3b1b3f81}{node} = ix + iy * ldx;
        tcur\_[\hyperlink{namespacevt_1_1config_a46e57c33bee1b9f1fa95455af86d80e0a576033c81e2ddd03d721ecef3b1b3f81}{node}] = 0.25 * (rhs\_[\hyperlink{namespacevt_1_1config_a46e57c33bee1b9f1fa95455af86d80e0a576033c81e2ddd03d721ecef3b1b3f81}{node}]
                              + told\_[node - 1] + told\_[node + 1]
                              + told\_[node - ldx] + told\_[node + ldx]);
      \}
    \}

    iter\_ += 1;

    std::copy(tcur\_.begin(), tcur\_.end(), told\_.begin());

    \textcolor{comment}{//}
    \textcolor{comment}{// Compute the maximum entries among the rows on this object}
    \textcolor{comment}{// We do not take into account the "ghost" entries}
    \textcolor{comment}{// as they may be "out of date".}
    \textcolor{comment}{//}

    \textcolor{keywordtype}{double} maxNorm = 0.0;

    \textcolor{keywordflow}{for} (\textcolor{keywordtype}{size\_t} iy = 1; iy <= numRowsPerObject\_; ++iy) \{
      \textcolor{keywordflow}{for} (\textcolor{keywordtype}{size\_t} ix = 1; ix <= numRowsPerObject\_; ++ix) \{
        \textcolor{keywordtype}{size\_t} \hyperlink{namespacevt_1_1config_a46e57c33bee1b9f1fa95455af86d80e0a576033c81e2ddd03d721ecef3b1b3f81}{node} = ix + iy * ldx;
        \textcolor{keywordtype}{double} val = tcur\_[\hyperlink{namespacevt_1_1config_a46e57c33bee1b9f1fa95455af86d80e0a576033c81e2ddd03d721ecef3b1b3f81}{node}];
        maxNorm = (maxNorm > std::fabs(val)) ? maxNorm : std::fabs(val);
      \}
    \}

    \textcolor{keyword}{auto} proxy = this->getCollectionProxy();
    \textcolor{keyword}{auto} cb = \hyperlink{namespacevt_a673b109e94c7bca58313504c83e1da94}{vt::theCB}()->\hyperlink{structvt_1_1pipe_1_1_pipe_manager_a73583be6260418b13ee66e56cdade2da}{makeSend}<
      LinearPb2DJacobi,ReduxMsg,&LinearPb2DJacobi::checkCompleteCB
    >(proxy(0,0));
    \textcolor{keyword}{auto} msg2 = vt::makeMessage<ReduxMsg>(maxNorm);
    proxy.\hyperlink{structvt_1_1objgroup_1_1proxy_1_1_proxy_a61273d407174fb496b4aed10ec6650bd}{reduce}<\hyperlink{structvt_1_1collective_1_1reduce_1_1operators_1_1_max_op}{vt::collective::MaxOp<double>}>(msg2.get(),cb);
  \}

  \textcolor{keyword}{struct }VecMsg : \hyperlink{structvt_1_1vrt_1_1collection_1_1_collection_message}{vt::CollectionMessage}<LinearPb2DJacobi> \{
    \textcolor{keyword}{using} MessageParentType = \hyperlink{structvt_1_1vrt_1_1collection_1_1_collection_message}{vt::CollectionMessage<LinearPb2DJacobi>}
      ;
    \hyperlink{message__serialize_8h_a54128d5338aaa9c918179826085e13d1}{vt\_msg\_serialize\_required}(); \textcolor{comment}{// stl vector}

    VecMsg() = \textcolor{keywordflow}{default};
    VecMsg(IndexType \textcolor{keyword}{const}& in\_index, \textcolor{keyword}{const} std::vector<double> &ref) :
      \hyperlink{namespacevt}{vt}::\hyperlink{namespacevt_ae7700e12f79c0fec16964aab84838428}{CollectionMessage}<LinearPb2DJacobi>(),
      from\_index(in\_index), val(ref)
    \{ \}

    \textcolor{keyword}{template} <\textcolor{keyword}{typename} Serializer>
    \textcolor{keywordtype}{void} \hyperlink{namespacecheckpoint_a075da4e7344cf037943362517e606c3a}{serialize}(Serializer& s) \{
      \hyperlink{namespacecheckpoint_a075da4e7344cf037943362517e606c3a}{MessageParentType::serialize}(s);
      s | from\_index;
      s | val;
    \}

    IndexType from\_index;
    std::vector<double> val;
  \};

  \textcolor{keywordtype}{void} exchange(VecMsg *msg) \{

    \textcolor{comment}{// Receive and treat the message from a neighboring object.}

    \textcolor{keywordflow}{if} (this->\hyperlink{namespacecheckpoint_1_1dispatch_1_1typeregistry_aa75ef84e9b63a687784360ded2d56fe4}{getIndex}().x() > msg->from\_index.x()) \{
      \textcolor{keyword}{const} \textcolor{keywordtype}{size\_t} ldx = numRowsPerObject\_ + 2;
      \textcolor{keywordflow}{for} (\textcolor{keywordtype}{size\_t} jy = 0; jy < msg->val.size(); ++jy) \{
        this->told\_[jy*ldx] = msg->val[jy];
      \}
      msgReceived\_ += 1;
    \}
    \textcolor{keywordflow}{else} \textcolor{keywordflow}{if} (this->\hyperlink{namespacecheckpoint_1_1dispatch_1_1typeregistry_aa75ef84e9b63a687784360ded2d56fe4}{getIndex}().x() < msg->from\_index.x()) \{
      \textcolor{keyword}{const} \textcolor{keywordtype}{size\_t} ldx = numRowsPerObject\_ + 2;
      \textcolor{keywordflow}{for} (\textcolor{keywordtype}{size\_t} jy = 0; jy < msg->val.size(); ++jy) \{
        this->told\_[numRowsPerObject\_ + 1 + jy*ldx] = msg->val[jy];
      \}
      msgReceived\_ += 1;
    \}
    \textcolor{keywordflow}{else} \textcolor{keywordflow}{if} (this->\hyperlink{namespacecheckpoint_1_1dispatch_1_1typeregistry_aa75ef84e9b63a687784360ded2d56fe4}{getIndex}().y() > msg->from\_index.y()) \{
      std::copy(msg->val.begin(), msg->val.end(), this->told\_.begin());
      msgReceived\_ += 1;
    \}
    \textcolor{keywordflow}{else} \textcolor{keywordflow}{if} (this->\hyperlink{namespacecheckpoint_1_1dispatch_1_1typeregistry_aa75ef84e9b63a687784360ded2d56fe4}{getIndex}().y() < msg->from\_index.y()) \{
      std::copy(msg->val.begin(), msg->val.end(),
                &this->told\_[(numRowsPerObject\_ + 1)*(numRowsPerObject\_ + 2)]);
      msgReceived\_ += 1;
    \}

    \textcolor{keywordflow}{if} (msgReceived\_ == totalReceive\_) \{
      msgReceived\_ = 0;
      doIteration();
    \}

  \}

  \textcolor{keywordtype}{void} doIter(BlankMsg *msg) \{

    \textcolor{comment}{//}
    \textcolor{comment}{// Treat the particular case of 1 object}
    \textcolor{comment}{// where no communication is needed.}
    \textcolor{comment}{// Without this treatment, the code would not iterate.}
    \textcolor{comment}{//}
    \textcolor{keywordflow}{if} (numObjsX\_*numObjsY\_ <= 1) \{
      doIteration();
      \textcolor{keywordflow}{return};
    \}
    \textcolor{comment}{//---------------------------------------}

    \textcolor{comment}{//}
    \textcolor{comment}{// Routine to send information to a neighboring object}
    \textcolor{comment}{//}

    \textcolor{keyword}{auto} proxy = this->getCollectionProxy();
    \textcolor{keyword}{auto} idx = this->\hyperlink{namespacecheckpoint_1_1dispatch_1_1typeregistry_aa75ef84e9b63a687784360ded2d56fe4}{getIndex}();
    \textcolor{keyword}{auto} \textcolor{keyword}{const} x = idx.x();
    \textcolor{keyword}{auto} \textcolor{keyword}{const} y = idx.y();

    \textcolor{keywordflow}{if} (x > 0) \{
      std::vector<double> tcopy(numRowsPerObject\_ + 2, 0.0);
      \textcolor{keywordflow}{for} (\textcolor{keywordtype}{size\_t} jy = 1; jy <= numRowsPerObject\_; ++jy)
        tcopy[jy] = told\_[1 + jy * (numRowsPerObject\_ + 2)];
      proxy(x-1, y).send<VecMsg, &LinearPb2DJacobi::exchange>(idx, tcopy);
    \}

    \textcolor{keywordflow}{if} (y > 0) \{
      std::vector<double> tcopy(numRowsPerObject\_ + 2, 0.0);
      \textcolor{keywordflow}{for} (\textcolor{keywordtype}{size\_t} jx = 1; jx <= numRowsPerObject\_; ++jx)
        tcopy[jx] = told\_[jx + (numRowsPerObject\_ + 2)];
      proxy(x, y-1).send<VecMsg, &LinearPb2DJacobi::exchange>(idx, tcopy);
    \}

    \textcolor{keywordflow}{if} (\textcolor{keywordtype}{size\_t}(x) < numObjsX\_ - 1) \{
      std::vector<double> tcopy(numRowsPerObject\_ + 2, 0.0);
      \textcolor{keywordflow}{for} (\textcolor{keywordtype}{size\_t} jy = 1; jy <= numRowsPerObject\_; ++jy) \{
        tcopy[jy] = told\_[numRowsPerObject\_ +
                          jy * (numRowsPerObject\_ + 2)];
      \}
      proxy(x+1, y).send<VecMsg, &LinearPb2DJacobi::exchange>(idx, tcopy);
    \}

    \textcolor{keywordflow}{if} (\textcolor{keywordtype}{size\_t}(y) < numObjsY\_ - 1) \{
      std::vector<double> tcopy(numRowsPerObject\_ + 2, 0.0);
      \textcolor{keywordflow}{for} (\textcolor{keywordtype}{size\_t} jx = 1; jx <= numRowsPerObject\_; ++jx)
        tcopy[jx] = told\_[jx + numRowsPerObject\_ * (numRowsPerObject\_ + 2)];
      proxy(x, y+1).send<VecMsg, &LinearPb2DJacobi::exchange>(idx, tcopy);
    \}

  \}


  \textcolor{keywordtype}{void} init() \{

    \textcolor{comment}{//--- Each object will work with (numRowsPerObject\_ + 2) unknowns}
    \textcolor{comment}{//--- or (numRowsPerObject\_ + 2) rows of the matrix}
    \textcolor{keywordtype}{size\_t} ldx = numRowsPerObject\_ + 2, ldy = ldx;

    \textcolor{keywordtype}{size\_t} vecSize = ldx * ldy;
    tcur\_.assign(vecSize, 0.0);
    told\_.assign(vecSize, 0.0);
    rhs\_.assign(vecSize, 0.0);

    \textcolor{comment}{//}
    \textcolor{comment}{// Set the initial vector to the values of}
    \textcolor{comment}{// a "high-frequency" function}
    \textcolor{comment}{//}

    \textcolor{keywordtype}{double} hx = 1.0 / (numRowsPerObject\_ * numObjsX\_ + 1.0);
    \textcolor{keywordtype}{double} hy = 1.0 / (numRowsPerObject\_ * numObjsY\_ + 1.0);

    \textcolor{keywordtype}{size\_t} maxNObjs = (size\_t) std::max(numObjsX\_, numObjsY\_);
    \textcolor{keywordtype}{int} nf = 3 * int(numRowsPerObject\_ * maxNObjs + 1) / 4;

    \textcolor{keyword}{auto} idx = this->\hyperlink{namespacecheckpoint_1_1dispatch_1_1typeregistry_aa75ef84e9b63a687784360ded2d56fe4}{getIndex}();

    \textcolor{keywordflow}{for} (\textcolor{keywordtype}{size\_t} iy = 0; iy < ldy; ++iy) \{
      \textcolor{keywordflow}{for} (\textcolor{keywordtype}{size\_t} ix = 0; ix < ldx; ++ix) \{
        \textcolor{keywordtype}{double} x0 = ( numRowsPerObject\_ * idx.x() + ix) * hx;
        \textcolor{keywordtype}{double} y0 = ( numRowsPerObject\_ * idx.y() + iy) * hy;
        \textcolor{keywordtype}{size\_t} \hyperlink{namespacevt_1_1config_a46e57c33bee1b9f1fa95455af86d80e0a576033c81e2ddd03d721ecef3b1b3f81}{node} = ix + iy * ldx;
        tcur\_[\hyperlink{namespacevt_1_1config_a46e57c33bee1b9f1fa95455af86d80e0a576033c81e2ddd03d721ecef3b1b3f81}{node}] = sin(nf * M\_PI * (x0 * x0 + y0 * y0));
      \}
    \}

    totalReceive\_ = 4;

    \textcolor{comment}{//}
    \textcolor{comment}{//--- The unknowns correspond to the interior nodes}
    \textcolor{comment}{//--- of a regular orthogonal grid on [0, 1] x [0, 1]}
    \textcolor{comment}{//--- The total number of grid points in X-direction is}
    \textcolor{comment}{//--- (numRowsPerObject\_ * numObjsX\_) + 2}
    \textcolor{comment}{//--- The total number of grid points in Y-direction is}
    \textcolor{comment}{//--- (numRowsPerObject\_ * numObjsY\_) + 2}
    \textcolor{comment}{//}
    \textcolor{keywordflow}{if} (idx.x() == 0) \{
      \textcolor{keywordflow}{for} (\textcolor{keywordtype}{size\_t} jy = 0; jy < ldy; ++jy)
        tcur\_[jy * ldy] = 0.0;
      totalReceive\_ -= 1;
    \}

    \textcolor{keywordflow}{if} (idx.y() == 0) \{
      \textcolor{keywordflow}{for} (\textcolor{keywordtype}{size\_t} jx = 0; jx < ldx; ++jx)
        tcur\_[jx] = 0.0;
      totalReceive\_ -= 1;
    \}

    \textcolor{keywordflow}{if} (numObjsX\_ == \textcolor{keywordtype}{size\_t}(idx.x()) + 1) \{
      \textcolor{keywordflow}{for} (\textcolor{keywordtype}{size\_t} jy = 0; jy < ldy; ++jy)
        tcur\_[jy * ldy + (ldx - 1)] = 0.0;
      totalReceive\_ -= 1;
    \}

    \textcolor{keywordflow}{if} (numObjsY\_ == \textcolor{keywordtype}{size\_t}(idx.y()) + 1) \{
      \textcolor{keywordflow}{for} (\textcolor{keywordtype}{size\_t} jx = 0; jx < ldx; ++jx)
        tcur\_[jx + (ldx - 1)*ldy] = 0.0;
      totalReceive\_ -= 1;
    \}

    std::copy(tcur\_.begin(), tcur\_.end(), told\_.begin());

  \}


  \textcolor{keywordtype}{void} init(LPMsg* msg) \{

    numObjsX\_ = msg->numXObjs;
    numObjsY\_ = msg->numYObjs;
    maxIter\_ = msg->numIter;
    objProxy\_ = msg->objProxy;

    \textcolor{comment}{// Initialize the starting vector}
    init();
  \}

\};

\textcolor{keywordtype}{bool} isWorkDone( \hyperlink{structvt_1_1objgroup_1_1proxy_1_1_proxy}{vt::objgroup::proxy::Proxy<NodeObj>} \textcolor{keyword}{const}& proxy)\{
  \textcolor{keyword}{auto} \textcolor{keyword}{const} this\_node = \hyperlink{namespacevt_a26551fe0e6e6a1371111df5b12c7e92c}{vt::theContext}()->\hyperlink{structvt_1_1ctx_1_1_context_a0d52c263ce8516546a67443d9a86fa5f}{getNode}();
  \textcolor{keywordflow}{return} proxy[this\_node].invoke<&NodeObj::isWorkFinished>();
\}

\textcolor{keywordtype}{int} main(\textcolor{keywordtype}{int} argc, \textcolor{keywordtype}{char}** argv) \{

  \textcolor{keywordtype}{size\_t} numX\_objs = default\_num\_objs;
  \textcolor{keywordtype}{size\_t} numY\_objs = default\_num\_objs;
  \textcolor{keywordtype}{size\_t} maxIter = 10;

  std::string name(argv[0]);

  \hyperlink{namespacevt_a580ef23e9a224fc0907da1a6db308b5b}{vt::initialize}(argc, argv);

  \hyperlink{namespacevt_a866da9d0efc19c0a1ce79e9e492f47e2}{vt::NodeType} this\_node = \hyperlink{namespacevt_a26551fe0e6e6a1371111df5b12c7e92c}{vt::theContext}()->\hyperlink{structvt_1_1ctx_1_1_context_a0d52c263ce8516546a67443d9a86fa5f}{getNode}();

  \textcolor{keywordflow}{if} (argc == 1) \{
    \textcolor{keywordflow}{if} (this\_node == 0) \{
      fmt::print(
        stderr, \textcolor{stringliteral}{"\{\}: using default arguments since none provided\(\backslash\)n"}, name
      );
    \}
  \} \textcolor{keywordflow}{else} \{
    \textcolor{keywordflow}{if} (argc == 3) \{
      numX\_objs = (size\_t) strtol(argv[1], \textcolor{keyword}{nullptr}, 10);
      numY\_objs = (size\_t) strtol(argv[2], \textcolor{keyword}{nullptr}, 10);
    \}
    \textcolor{keywordflow}{else} \textcolor{keywordflow}{if} (argc == 4) \{
      numX\_objs = (size\_t) strtol(argv[1], \textcolor{keyword}{nullptr}, 10);
      numY\_objs = (size\_t) strtol(argv[2], \textcolor{keyword}{nullptr}, 10);
      maxIter = (size\_t) strtol(argv[3], \textcolor{keyword}{nullptr}, 10);
    \}
    \textcolor{keywordflow}{else} \{
      fmt::print(
        stderr, \textcolor{stringliteral}{"usage: \{\} <num-objects-X-direction> <num-objects-Y-direction> <maxiter>\(\backslash\)n"},
        name
      );
      \textcolor{keywordflow}{return} 1;
    \}
  \}

  \textcolor{comment}{/* --- Print information about the simulation */}

  \textcolor{keywordflow}{if} (this\_node == 0) \{
    fmt::print(
      stdout, \textcolor{stringliteral}{"\(\backslash\)n - Solve the linear system for the Laplacian with homogeneous Dirichlet"}
      \textcolor{stringliteral}{" on [0, 1] x [0, 1]\(\backslash\)n"}
    );
    fmt::print(stdout, \textcolor{stringliteral}{" - Second-order centered finite difference\(\backslash\)n"});
    fmt::print(
      stdout, \textcolor{stringliteral}{" - Uniform grid with (\{\} x \{\} = \{\}) points in the x-direction and "}
      \textcolor{stringliteral}{" (\{\} x \{\} = \{\}) points in the y-direction\(\backslash\)n"},
      numX\_objs, default\_nrow\_object, numX\_objs * default\_nrow\_object,
      numY\_objs, default\_nrow\_object, numY\_objs * default\_nrow\_object
    );
    fmt::print(stdout, \textcolor{stringliteral}{" - Maximum number of iterations \{\}\(\backslash\)n"}, maxIter);
    fmt::print(stdout, \textcolor{stringliteral}{" - Convergence tolerance \{\}\(\backslash\)n"}, default\_tol);
    fmt::print(stdout, \textcolor{stringliteral}{"\(\backslash\)n"});
  \}

  \textcolor{comment}{// Object group of all nodes that take part in computation}
  \textcolor{comment}{// Used to determine whether the computation is finished}
  \textcolor{keyword}{auto} grp\_proxy = \hyperlink{namespacevt_a833f0115b692f578167cbd88e30d39c5}{vt::theObjGroup}()->\hyperlink{structvt_1_1objgroup_1_1_obj_group_manager_a50e0d8498d142a05773ed5a046803b2b}{makeCollective}<NodeObj>(\textcolor{stringliteral}{"
      examples\_jacobi2d"});

  \textcolor{comment}{// Create the decomposition into objects}
  \textcolor{keyword}{using} BaseIndexType = \textcolor{keyword}{typename} \hyperlink{structvt_1_1index_1_1_dense_index_array_a36698427e28045290d1fb072573275ec}{vt::Index2D::DenseIndexType};
  \textcolor{keyword}{auto} range = \hyperlink{namespacevt_a3bab786053b74a3d856fff1412ffa73a}{vt::Index2D}(
    static\_cast<BaseIndexType>(numX\_objs),
    static\_cast<BaseIndexType>(numY\_objs)
  );

  \textcolor{keyword}{auto} col\_proxy = vt::makeCollection<LinearPb2DJacobi>(\textcolor{stringliteral}{"examples\_jacobi2d"})
    .bounds(range)
    .bulkInsert()
    .wait();

  \hyperlink{namespacevt_a2fc4ef34f30b49a1781d765804bfadbb}{vt::runInEpochCollective}([col\_proxy, grp\_proxy, numX\_objs, numY\_objs, maxIter] \{
    col\_proxy.broadcastCollective<LinearPb2DJacobi::LPMsg, &LinearPb2DJacobi::init>(
      numX\_objs, numY\_objs, maxIter, grp\_proxy
    );
  \});

  \textcolor{keywordflow}{while} (!isWorkDone(grp\_proxy)) \{
    \hyperlink{namespacevt_a2fc4ef34f30b49a1781d765804bfadbb}{vt::runInEpochCollective}([col\_proxy] \{
      col\_proxy.broadcastCollective<
        LinearPb2DJacobi::BlankMsg, &LinearPb2DJacobi::doIter>();
    \});

    \hyperlink{namespacevt_a3799d6b51fb71d7b6505760aad457e61}{vt::thePhase}()->\hyperlink{structvt_1_1phase_1_1_phase_manager_ac94bfde4a103cbdec426542191c41092}{nextPhaseCollective}();
  \}

  \hyperlink{namespacevt_a540d90dbd6e97b69f1dcbc9ee9314cff}{vt::finalize}();

  \textcolor{keywordflow}{return} 0;

\}
\end{DoxyCodeInclude}
\hypertarget{reduce-integral-example}{}\subsection{Example Integral Reduction Program}\label{reduce-integral-example}
The full code for this example can be found here\+: {\ttfamily examples/collection/reduce\+\_\+integral.\+cc}.


\begin{DoxyCodeInclude}

\textcolor{comment}{//}
\textcolor{comment}{// This code computes a composite trapezoidal rule for the integral}
\textcolor{comment}{// \(\backslash\)int\_\{0\}^\{1\} f(x) dx}
\textcolor{comment}{//}
\textcolor{comment}{// The function 'f' is defined in the code.}
\textcolor{comment}{//}
\textcolor{comment}{// The interval [0, 1] is broken into uniform sub-intervals of length}
\textcolor{comment}{//}
\textcolor{comment}{// H = 1.0 / (# of objects)}
\textcolor{comment}{//}
\textcolor{comment}{// Each sub-interval [xi, xi + H] is broken into smaller parts of length}
\textcolor{comment}{//}
\textcolor{comment}{// h = H / (# of parts per object)}
\textcolor{comment}{//}


\textcolor{preprocessor}{#include <\hyperlink{transport_8h}{vt/transport.h}>}
\textcolor{preprocessor}{#include <cmath>}

\textcolor{keyword}{static} constexpr std::size\_t \textcolor{keyword}{const} default\_nparts\_object = 8;
\textcolor{keyword}{static} constexpr std::size\_t \textcolor{keyword}{const} default\_num\_objs = 4;
\textcolor{keyword}{static} constexpr std::size\_t \textcolor{keyword}{const} \hyperlink{namespacevt_1_1config_a724e99568632cc0fe9c456c2230270d0a9b54bf70f7b7fe7cfb6f3da0e167ae27}{verbose} = 1;
\textcolor{keyword}{static} constexpr \hyperlink{namespacevt_a866da9d0efc19c0a1ce79e9e492f47e2}{vt::NodeType} \textcolor{keyword}{const} reduce\_root\_node = 0;

\textcolor{keyword}{static} \textcolor{keywordtype}{bool} root\_reduce\_finished = \textcolor{keyword}{false};
\textcolor{keyword}{static} \textcolor{keywordtype}{double} exactIntegral = 0.0;


\textcolor{comment}{//}
\textcolor{comment}{// Function 'f' to integrate over [0, 1]}
\textcolor{comment}{//}


\textcolor{keywordtype}{double} f(\textcolor{keywordtype}{double} x) \{
  \textcolor{comment}{//----}
  \textcolor{comment}{//exactIntegral = 1.0 / 3.0;}
  \textcolor{comment}{//return x*x;}
  \textcolor{comment}{//----}
  exactIntegral = M\_2\_PI;
  \textcolor{keywordflow}{return} sin(M\_PI * x);
\}

\textcolor{keyword}{using} ReduceMsg = \hyperlink{structvt_1_1collective_1_1reduce_1_1operators_1_1_reduce_t_msg}{vt::collective::ReduceTMsg<double>};

\textcolor{keyword}{struct }Integration1D : \hyperlink{structvt_1_1vrt_1_1collection_1_1_collection}{vt::Collection}<Integration1D, vt::Index1D> \{

\textcolor{keyword}{private}:

  \textcolor{keywordtype}{size\_t} numObjs\_ = default\_num\_objs;
  \textcolor{keywordtype}{size\_t} numPartsPerObject\_ = default\_nparts\_object;

\textcolor{keyword}{public}:

  \textcolor{keyword}{explicit} Integration1D()
    : numObjs\_(default\_num\_objs),
      numPartsPerObject\_(default\_nparts\_object)
  \{ \}

  \textcolor{keyword}{struct }CheckIntegral \{

    \textcolor{keywordtype}{void} operator()(ReduceMsg* msg) \{
      fmt::print(\textcolor{stringliteral}{" >> The integral over [0, 1] is \{\}\(\backslash\)n"}, msg->getConstVal());
      fmt::print(
        \textcolor{stringliteral}{" >> The absolute error is \{\}\(\backslash\)n"},
        std::fabs(msg->getConstVal() - exactIntegral)
      );
      \textcolor{comment}{//}
      \textcolor{comment}{// Set the 'root\_reduce\_finished' variable to true.}
      \textcolor{comment}{//}
      root\_reduce\_finished = \textcolor{keyword}{true};
    \}

  \};

  \textcolor{keyword}{struct }InitMsg : \hyperlink{structvt_1_1vrt_1_1collection_1_1_collection_message}{vt::CollectionMessage}<Integration1D> \{

    \textcolor{keywordtype}{size\_t} numObjects = 0;
    \textcolor{keywordtype}{size\_t} nIntervalPerObject = 0;

    InitMsg(\textcolor{keyword}{const} \textcolor{keywordtype}{size\_t} nobjs, \textcolor{keyword}{const} \textcolor{keywordtype}{size\_t} nint)
      : numObjects(nobjs), nIntervalPerObject(nint)
    \{ \}

  \};

  \textcolor{keywordtype}{void} compute(InitMsg *msg) \{

    numObjs\_ = msg->numObjects;
    numPartsPerObject\_ = msg->nIntervalPerObject;

    \textcolor{comment}{//}
    \textcolor{comment}{// Compute the integral with the trapezoidal rule}
    \textcolor{comment}{// over the interval [a, b]}
    \textcolor{comment}{//}
    \textcolor{comment}{// a = numPartsPerObject\_ * (IndexID) * h}
    \textcolor{comment}{//}
    \textcolor{comment}{// b = a + numPartsPerObject\_ * h}
    \textcolor{comment}{//}
    \textcolor{comment}{// where h = 1.0 / (numPartsPerObject\_ * numObjs\_ )}
    \textcolor{comment}{//}
    \textcolor{comment}{// Since 0 <= (IndexID) < numObjs\_, the union of these intervals}
    \textcolor{comment}{// covers [0, 1]}
    \textcolor{comment}{//}

    \textcolor{keywordtype}{double} h = 1.0 / (numPartsPerObject\_ * numObjs\_ );
    \textcolor{keywordtype}{double} quadsum = 0.0;

    \textcolor{keywordtype}{double} a = numPartsPerObject\_ * \hyperlink{namespacecheckpoint_1_1dispatch_1_1typeregistry_aa75ef84e9b63a687784360ded2d56fe4}{getIndex}().x() * h;

    \textcolor{comment}{//}
    \textcolor{comment}{// Apply composite trapezoidal rule over}
    \textcolor{comment}{//}
    \textcolor{comment}{// [a, a + numPartsPerObject\_ * h]}
    \textcolor{comment}{//}
    \textcolor{keywordflow}{for} (\textcolor{keywordtype}{size\_t} ii = 0; ii < numPartsPerObject\_; ++ii) \{
      \textcolor{keywordtype}{double} x0 = a + ii * h;
      \textcolor{comment}{/* --- Trapeze Quadrature Rule over the interval [x0, x0+h] */}
      quadsum += 0.5 * h * ( f(x0) + f(x0+h) );
    \}

    \textcolor{keywordflow}{if} (verbose > 0) \{
      \textcolor{keywordtype}{double} b = a + numPartsPerObject\_ * h;
      fmt::print(
        \textcolor{stringliteral}{" Interval [\{\}, \{\}], on node \{\} & object \{\}, "}
        \textcolor{stringliteral}{"has integral \{\}.\(\backslash\)n"}, a, b, \hyperlink{namespacevt_a26551fe0e6e6a1371111df5b12c7e92c}{vt::theContext}()->getNode(),
        \hyperlink{namespacecheckpoint_1_1dispatch_1_1typeregistry_aa75ef84e9b63a687784360ded2d56fe4}{getIndex}(), quadsum
      );
    \}

    \textcolor{comment}{//}
    \textcolor{comment}{// Reduce the partial sum to get the integral over [0, 1]}
    \textcolor{comment}{//}

    \textcolor{keyword}{auto} proxy = this->getCollectionProxy();
    \textcolor{keyword}{auto} msgCB = vt::makeMessage<ReduceMsg>(quadsum);
    \textcolor{keyword}{auto} cback = \hyperlink{namespacevt_a673b109e94c7bca58313504c83e1da94}{vt::theCB}()->\hyperlink{structvt_1_1pipe_1_1_pipe_manager_a73583be6260418b13ee66e56cdade2da}{makeSend}<CheckIntegral>(reduce\_root\_node);
    proxy.\hyperlink{structvt_1_1objgroup_1_1proxy_1_1_proxy_a61273d407174fb496b4aed10ec6650bd}{reduce}<\hyperlink{structvt_1_1collective_1_1reduce_1_1operators_1_1_plus_op}{vt::collective::PlusOp<double>}>(msgCB.get(),cback);
  \}

\};


\textcolor{keywordtype}{int} main(\textcolor{keywordtype}{int} argc, \textcolor{keywordtype}{char}** argv) \{

  \textcolor{keywordtype}{size\_t} num\_objs = default\_num\_objs;
  \textcolor{keywordtype}{size\_t} numIntPerObject = default\_nparts\_object;

  std::string name(argv[0]);

  \hyperlink{namespacevt_a580ef23e9a224fc0907da1a6db308b5b}{vt::initialize}(argc, argv);

  \hyperlink{namespacevt_a866da9d0efc19c0a1ce79e9e492f47e2}{vt::NodeType} this\_node = \hyperlink{namespacevt_a26551fe0e6e6a1371111df5b12c7e92c}{vt::theContext}()->\hyperlink{structvt_1_1ctx_1_1_context_a0d52c263ce8516546a67443d9a86fa5f}{getNode}();
  \hyperlink{namespacevt_a866da9d0efc19c0a1ce79e9e492f47e2}{vt::NodeType} num\_nodes = \hyperlink{namespacevt_a26551fe0e6e6a1371111df5b12c7e92c}{vt::theContext}()->
      \hyperlink{structvt_1_1ctx_1_1_context_a7f41071aadf6d5fa9e1b6c703c5ff19d}{getNumNodes}();

  \textcolor{keywordflow}{if} (argc == 1) \{
    \textcolor{keywordflow}{if} (this\_node == 0) \{
      fmt::print(
        stderr, \textcolor{stringliteral}{"\{\}: using default arguments since none provided\(\backslash\)n"}, name
      );
    \}
    num\_objs = default\_num\_objs * num\_nodes;
  \} \textcolor{keywordflow}{else} \{
    \textcolor{keywordflow}{if} (argc == 2) \{
      num\_objs = (size\_t) strtol(argv[1], \textcolor{keyword}{nullptr}, 10);
    \}
    \textcolor{keywordflow}{else} \textcolor{keywordflow}{if} (argc == 3) \{
      num\_objs = (size\_t) strtol(argv[1], \textcolor{keyword}{nullptr}, 10);
      numIntPerObject = (size\_t) strtol(argv[2], \textcolor{keyword}{nullptr}, 10);
    \}
    \textcolor{keywordflow}{else} \{
      fmt::print(
        stderr,
        \textcolor{stringliteral}{"usage: \{\} <num-objects> <num-interval-per-object>\(\backslash\)n"}, name
      );
      \textcolor{keywordflow}{return} 1;
    \}
  \}

  \textcolor{comment}{//}
  \textcolor{comment}{// Create the interval decomposition into objects}
  \textcolor{comment}{//}
  \textcolor{keyword}{using} BaseIndexType = \textcolor{keyword}{typename} \hyperlink{structvt_1_1index_1_1_dense_index_array_a36698427e28045290d1fb072573275ec}{vt::Index1D::DenseIndexType};
  \textcolor{keyword}{auto} range = \hyperlink{namespacevt_a5540efc78234273e1796fb003fe4d234}{vt::Index1D}(static\_cast<BaseIndexType>(num\_objs));

  \textcolor{keyword}{auto} proxy = vt::makeCollection<Integration1D>(\textcolor{stringliteral}{"examples\_reduce\_integral"})
    .bounds(range)
    .bulkInsert()
    .wait();

  \hyperlink{namespacevt_a2fc4ef34f30b49a1781d765804bfadbb}{vt::runInEpochCollective}([proxy, num\_objs, numIntPerObject]\{
    proxy.broadcastCollective<Integration1D::InitMsg, &Integration1D::compute>(
      num\_objs, numIntPerObject
    );
  \});

  \textcolor{comment}{// Add something like this to validate the reduction.}
  \textcolor{comment}{// Create the variable root\_reduce\_finished as a static variable,}
  \textcolor{comment}{// which is only checked on one node.}
  \textcolor{keywordflow}{if} (\hyperlink{namespacevt_a26551fe0e6e6a1371111df5b12c7e92c}{vt::theContext}()->\hyperlink{structvt_1_1ctx_1_1_context_a0d52c263ce8516546a67443d9a86fa5f}{getNode}() == reduce\_root\_node) \{
    \hyperlink{config__assert_8h_a60b04695e9ed43131da50386a2ced779}{vtAssertExpr}(root\_reduce\_finished == \textcolor{keyword}{true});
  \}

  \hyperlink{namespacevt_a540d90dbd6e97b69f1dcbc9ee9314cff}{vt::finalize}();

  \textcolor{keywordflow}{return} 0;

\}
\end{DoxyCodeInclude}
