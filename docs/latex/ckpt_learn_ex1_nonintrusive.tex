\begin{DoxyAttention}{Attention}
All Non-\/\+Intrusive serialize methods {\bfseries M\+U\+ST} be placed in the namespace of type which they serialize.
\end{DoxyAttention}
The full code for this {\itshape checkpoint} example can be found here\+: {\ttfamily examples/checkpoint\+\_\+example\+\_\+1\+\_\+nonintrusive.\+cc}


\begin{DoxyCodeInclude}

\textcolor{preprocessor}{#include <\hyperlink{checkpoint_8h}{checkpoint/checkpoint.h}>}

\textcolor{preprocessor}{#include <cstdio>}

\textcolor{comment}{// \(\backslash\)brief Namespace containing type which will be serialized}
\textcolor{keyword}{namespace }\hyperlink{namespacecheckpoint}{checkpoint} \{ \textcolor{keyword}{namespace }nonintrusive \{ \textcolor{keyword}{namespace }examples \{

\textcolor{comment}{// \(\backslash\)brief Simple structure with three variables of built-in types}
\textcolor{keyword}{struct }MyTest3 \{
  \textcolor{keywordtype}{int} a = 1, b = 2 , c = 3;

  \textcolor{comment}{// \(\backslash\)brief Printing function unto the standard display}
  \textcolor{keywordtype}{void} print() \{
    printf(\textcolor{stringliteral}{"MyTest3: a=%d, b=%d, c=%d\(\backslash\)n"}, a, b, c);
  \}
\};

\}\}\} \textcolor{comment}{// end namespace checkpoint::nonintrusive::examples}

\textcolor{comment}{// \(\backslash\)brief Function to serialize the MyTest3 structure.}
\textcolor{comment}{// In Non-Intrusive way, this function needs to be placed in the namespace}
\textcolor{comment}{// of the type which will be serialized.}
\textcolor{keyword}{namespace }\hyperlink{namespacecheckpoint}{checkpoint} \{ \textcolor{keyword}{namespace }nonintrusive \{ \textcolor{keyword}{namespace }examples \{
  \textcolor{comment}{// \(\backslash\)brief Templated function for serializing/deserializing}
  \textcolor{comment}{// a variable of type `MyTest3`. Non-nonintrusive version of the function}
  \textcolor{comment}{// placed outside of `MyTest3` structure.}
  \textcolor{comment}{//}
  \textcolor{comment}{// \(\backslash\)tparam <Serializer> The type of serializer depending on the pass}
  \textcolor{comment}{// \(\backslash\)param[in,out] s the serializer for traversing this class}
  \textcolor{comment}{//}
  \textcolor{comment}{// \(\backslash\)note The serialize method is typically called three times when}
  \textcolor{comment}{// (de-)serializing to a byte buffer:}
  \textcolor{comment}{//}
  \textcolor{comment}{// 1) Sizing: The first time its called, it sizes all the data it recursively}
  \textcolor{comment}{// traverses to generate a final size for the buffer.}
  \textcolor{comment}{//}
  \textcolor{comment}{// 2) Packing: As the traversal occurs, it copies the data traversed to the}
  \textcolor{comment}{// byte buffer in the appropriate location.}
  \textcolor{comment}{//}
  \textcolor{comment}{// 3) Unpacking: As the byte buffer is traversed, it extracts the bytes from}
  \textcolor{comment}{// the buffer to recursively reconstruct the types and setup the class members.}
  \textcolor{comment}{//}
  \textcolor{keyword}{template} <\textcolor{keyword}{typename} Serializer>
  \textcolor{keywordtype}{void} \hyperlink{namespacecheckpoint_a075da4e7344cf037943362517e606c3a}{serialize}(Serializer& s, MyTest3& my\_test3) \{
    \textcolor{comment}{// a, b, c - variable of type `int` (built-in type)}
    s | my\_test3.a;
    s | my\_test3.b;
    s | my\_test3.c;
  \}

\}\}\} \textcolor{comment}{// end namespace checkpoint::nonintrusive::examples}

\textcolor{keywordtype}{int} main(\textcolor{keywordtype}{int}, \textcolor{keywordtype}{char}**) \{
  \textcolor{keyword}{using namespace }magistrate::nonintrusive::examples;

  \textcolor{comment}{// Define a variable of custom type `MyTest3`}
  MyTest3 my\_test3;
  my\_test3.print();

  \textcolor{comment}{// Call the serialization routine for the variable `my\_test3`}
  \textcolor{comment}{// The output is a unique pointer: `std::unique\_ptr<SerializedInfo>`}
  \textcolor{comment}{// (defined in `src/checkpoint\_api.h`)}
  \textcolor{keyword}{auto} ret = \hyperlink{namespacecheckpoint_a075da4e7344cf037943362517e606c3a}{checkpoint::serialize}(my\_test3);

  \textcolor{keyword}{auto} \textcolor{keyword}{const}& buf = ret->getBuffer();
  \textcolor{keyword}{auto} \textcolor{keyword}{const}& buf\_size = ret->getSize();

  \textcolor{comment}{// Print the buffer address and its size}
  printf(\textcolor{stringliteral}{"ptr=%p, size=%ld\(\backslash\)n"}, static\_cast<void*>(buf), buf\_size);

  \textcolor{comment}{// Deserialize the variable `my\_test3` from the serialized buffer}
  \textcolor{keyword}{auto} t = checkpoint::deserialize<MyTest3>(buf);

  \textcolor{comment}{// Display the de-serialized data}
  t->print();

  \textcolor{keywordflow}{return} 0;
\}

\end{DoxyCodeInclude}
