The full code for this {\itshape checkpoint} example can be found here\+: {\ttfamily examples/checkpoint\+\_\+example\+\_\+1.\+cc}


\begin{DoxyCodeInclude}{0}
\DoxyCodeLine{}
\DoxyCodeLine{\textcolor{preprocessor}{\#include <\mbox{\hyperlink{checkpoint_8h}{checkpoint/checkpoint.h}}>}}
\DoxyCodeLine{}
\DoxyCodeLine{\textcolor{preprocessor}{\#include <cstdio>}}
\DoxyCodeLine{}
\DoxyCodeLine{\textcolor{keyword}{namespace }\mbox{\hyperlink{namespacecheckpoint}{checkpoint}} \{ \textcolor{keyword}{namespace }intrusive \{ \textcolor{keyword}{namespace }examples \{}
\DoxyCodeLine{}
\DoxyCodeLine{\textcolor{comment}{// \(\backslash\)struct MyTest}}
\DoxyCodeLine{\textcolor{comment}{// \(\backslash\)brief Simple structure with two variables of built-\/in types}}
\DoxyCodeLine{\textcolor{keyword}{struct }MyTest \{}
\DoxyCodeLine{  \textcolor{keywordtype}{int} a = 29, b = 31;}
\DoxyCodeLine{}
\DoxyCodeLine{  \textcolor{comment}{// \(\backslash\)brief Default constructor}}
\DoxyCodeLine{  \textcolor{comment}{//}}
\DoxyCodeLine{  \textcolor{comment}{// The reconstruction strategy is required for deserialization. A default}}
\DoxyCodeLine{  \textcolor{comment}{// constructor is one of the reconstruction strategies that checkpoint will}}
\DoxyCodeLine{  \textcolor{comment}{// look for.}}
\DoxyCodeLine{  MyTest() = \textcolor{keywordflow}{default};}
\DoxyCodeLine{}
\DoxyCodeLine{  \textcolor{comment}{// \(\backslash\)brief Constructor with two parameters}}
\DoxyCodeLine{  \textcolor{comment}{//}}
\DoxyCodeLine{  \textcolor{comment}{// \(\backslash\)param[in] Initial value for `a`}}
\DoxyCodeLine{  \textcolor{comment}{// \(\backslash\)param[in] Initial value for `b`}}
\DoxyCodeLine{  \textcolor{comment}{//}}
\DoxyCodeLine{  MyTest(\textcolor{keywordtype}{int} ai, \textcolor{keywordtype}{int} bi) : a(ai), b(bi) \{ \};}
\DoxyCodeLine{}
\DoxyCodeLine{  \textcolor{comment}{// \(\backslash\)brief Printing function unto the standard display}}
\DoxyCodeLine{  \textcolor{keywordtype}{void} print() \{}
\DoxyCodeLine{    printf(\textcolor{stringliteral}{"{}MyTest: a=\%d, b=\%d\(\backslash\)n"{}}, a, b);}
\DoxyCodeLine{  \}}
\DoxyCodeLine{}
\DoxyCodeLine{  \textcolor{comment}{// \(\backslash\)brief Templated function for serializing/deserializing}}
\DoxyCodeLine{  \textcolor{comment}{// a variable of type `MyTest`}}
\DoxyCodeLine{  \textcolor{comment}{//}}
\DoxyCodeLine{  \textcolor{comment}{// \(\backslash\)tparam <Serializer> The type of serializer depending on the pass}}
\DoxyCodeLine{  \textcolor{comment}{// \(\backslash\)param[in,out] s the serializer for traversing this class}}
\DoxyCodeLine{  \textcolor{comment}{//}}
\DoxyCodeLine{  \textcolor{comment}{// \(\backslash\)note The serialize method is typically called three times when}}
\DoxyCodeLine{  \textcolor{comment}{// (de-\/)serializing to a byte buffer:}}
\DoxyCodeLine{  \textcolor{comment}{//}}
\DoxyCodeLine{  \textcolor{comment}{// 1) Sizing: The first time its called, it sizes all the data it recursively}}
\DoxyCodeLine{  \textcolor{comment}{// traverses to generate a final size for the buffer.}}
\DoxyCodeLine{  \textcolor{comment}{//}}
\DoxyCodeLine{  \textcolor{comment}{// 2) Packing: As the traversal occurs, it copies the data traversed to the}}
\DoxyCodeLine{  \textcolor{comment}{// byte buffer in the appropriate location.}}
\DoxyCodeLine{  \textcolor{comment}{//}}
\DoxyCodeLine{  \textcolor{comment}{// 3) Unpacking: As the byte buffer is traversed, it extracts the bytes from}}
\DoxyCodeLine{  \textcolor{comment}{// the buffer to recursively reconstruct the types and setup the class members.}}
\DoxyCodeLine{  \textcolor{comment}{//}}
\DoxyCodeLine{  \textcolor{keyword}{template} <\textcolor{keyword}{typename} Serializer>}
\DoxyCodeLine{  \textcolor{keywordtype}{void} \mbox{\hyperlink{namespacecheckpoint_a075da4e7344cf037943362517e606c3a}{serialize}}(Serializer\& s) \{}
\DoxyCodeLine{    printf(\textcolor{stringliteral}{"{}MyTest serialize\(\backslash\)n"{}});}
\DoxyCodeLine{    \textcolor{comment}{//}}
\DoxyCodeLine{    \textcolor{comment}{// a = variable of type `int` (built-\/in type)}}
\DoxyCodeLine{    \textcolor{comment}{//}}
\DoxyCodeLine{    s | a;}
\DoxyCodeLine{    \textcolor{comment}{//}}
\DoxyCodeLine{    \textcolor{comment}{// b = variable of type `int` (built-\/in type)}}
\DoxyCodeLine{    \textcolor{comment}{//}}
\DoxyCodeLine{    s | b;}
\DoxyCodeLine{  \}}
\DoxyCodeLine{\};}
\DoxyCodeLine{}
\DoxyCodeLine{\}\}\} \textcolor{comment}{// end namespace checkpoint::intrusive::examples}}
\DoxyCodeLine{}
\DoxyCodeLine{\textcolor{keywordtype}{int} main(\textcolor{keywordtype}{int}, \textcolor{keywordtype}{char}**) \{}
\DoxyCodeLine{  \textcolor{keyword}{using namespace }magistrate::intrusive::examples;}
\DoxyCodeLine{}
\DoxyCodeLine{  \textcolor{comment}{// Define a variable of custom type `MyTest`}}
\DoxyCodeLine{  MyTest my\_test\_inst(11, 12);}
\DoxyCodeLine{  my\_test\_inst.print();}
\DoxyCodeLine{}
\DoxyCodeLine{  \textcolor{comment}{// Call the serialization routine for the variable `my\_test\_inst`}}
\DoxyCodeLine{  \textcolor{comment}{// The output is a unique pointer: `std::unique\_ptr<SerializedInfo>`}}
\DoxyCodeLine{  \textcolor{comment}{// (defined in `src/checkpoint\_api.h`)}}
\DoxyCodeLine{  \textcolor{keyword}{auto} ret = \mbox{\hyperlink{namespacecheckpoint_a075da4e7344cf037943362517e606c3a}{checkpoint::serialize}}(my\_test\_inst);}
\DoxyCodeLine{}
\DoxyCodeLine{  \{}
\DoxyCodeLine{    \textcolor{comment}{// Display information about the serialization "{}message"{}}}
\DoxyCodeLine{    \textcolor{keyword}{auto} \textcolor{keyword}{const}\& buf = ret-\/>getBuffer();}
\DoxyCodeLine{    \textcolor{keyword}{auto} \textcolor{keyword}{const}\& buf\_size = ret-\/>getSize();}
\DoxyCodeLine{    printf(\textcolor{stringliteral}{"{}ptr=\%p, size=\%ld\(\backslash\)n"{}}, \textcolor{keyword}{static\_cast<}\textcolor{keywordtype}{void}*\textcolor{keyword}{>}(buf), buf\_size);}
\DoxyCodeLine{  \}}
\DoxyCodeLine{}
\DoxyCodeLine{  \textcolor{comment}{// De-\/serialization call to create a new unique pointer to `MyTest`}}
\DoxyCodeLine{  \textcolor{keyword}{auto} t = checkpoint::deserialize<MyTest>(ret-\/>getBuffer());}
\DoxyCodeLine{}
\DoxyCodeLine{  \textcolor{comment}{// Display the result}}
\DoxyCodeLine{  t-\/>print();}
\DoxyCodeLine{}
\DoxyCodeLine{  \textcolor{keywordflow}{return} 0;}
\DoxyCodeLine{\}}
\DoxyCodeLine{}

\end{DoxyCodeInclude}
