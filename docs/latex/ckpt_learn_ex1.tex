The full code for this {\itshape checkpoint} example can be found here\+: {\ttfamily examples/checkpoint\+\_\+example\+\_\+1.\+cc}


\begin{DoxyCodeInclude}

\textcolor{preprocessor}{#include <\hyperlink{checkpoint_8h}{checkpoint/checkpoint.h}>}

\textcolor{preprocessor}{#include <cstdio>}

\textcolor{keyword}{namespace }magistrate \{ \textcolor{keyword}{namespace }intrusive \{ \textcolor{keyword}{namespace }examples \{

\textcolor{comment}{// \(\backslash\)struct MyTest}
\textcolor{comment}{// \(\backslash\)brief Simple structure with two variables of built-in types}
\textcolor{keyword}{struct }MyTest \{
  \textcolor{keywordtype}{int} a = 29, b = 31;

  \textcolor{comment}{// \(\backslash\)brief Default constructor}
  \textcolor{comment}{//}
  \textcolor{comment}{// The reconstruction strategy is required for deserialization. A default}
  \textcolor{comment}{// constructor is one of the reconstruction strategies that checkpoint will}
  \textcolor{comment}{// look for.}
  MyTest() = \textcolor{keywordflow}{default};

  \textcolor{comment}{// \(\backslash\)brief Constructor with two parameters}
  \textcolor{comment}{//}
  \textcolor{comment}{// \(\backslash\)param[in] Initial value for `a`}
  \textcolor{comment}{// \(\backslash\)param[in] Initial value for `b`}
  \textcolor{comment}{//}
  MyTest(\textcolor{keywordtype}{int} ai, \textcolor{keywordtype}{int} bi) : a(ai), b(bi) \{ \};

  \textcolor{comment}{// \(\backslash\)brief Printing function unto the standard display}
  \textcolor{keywordtype}{void} print() \{
    printf(\textcolor{stringliteral}{"MyTest: a=%d, b=%d\(\backslash\)n"}, a, b);
  \}

  \textcolor{comment}{// \(\backslash\)brief Templated function for serializing/deserializing}
  \textcolor{comment}{// a variable of type `MyTest`}
  \textcolor{comment}{//}
  \textcolor{comment}{// \(\backslash\)tparam <Serializer> The type of serializer depending on the pass}
  \textcolor{comment}{// \(\backslash\)param[in,out] s the serializer for traversing this class}
  \textcolor{comment}{//}
  \textcolor{comment}{// \(\backslash\)note The serialize method is typically called three times when}
  \textcolor{comment}{// (de-)serializing to a byte buffer:}
  \textcolor{comment}{//}
  \textcolor{comment}{// 1) Sizing: The first time its called, it sizes all the data it recursively}
  \textcolor{comment}{// traverses to generate a final size for the buffer.}
  \textcolor{comment}{//}
  \textcolor{comment}{// 2) Packing: As the traversal occurs, it copies the data traversed to the}
  \textcolor{comment}{// byte buffer in the appropriate location.}
  \textcolor{comment}{//}
  \textcolor{comment}{// 3) Unpacking: As the byte buffer is traversed, it extracts the bytes from}
  \textcolor{comment}{// the buffer to recursively reconstruct the types and setup the class members.}
  \textcolor{comment}{//}
  \textcolor{keyword}{template} <\textcolor{keyword}{typename} Serializer>
  \textcolor{keywordtype}{void} \hyperlink{namespacecheckpoint_a075da4e7344cf037943362517e606c3a}{serialize}(Serializer& s) \{
    printf(\textcolor{stringliteral}{"MyTest serialize\(\backslash\)n"});
    \textcolor{comment}{//}
    \textcolor{comment}{// a = variable of type `int` (built-in type)}
    \textcolor{comment}{//}
    s | a;
    \textcolor{comment}{//}
    \textcolor{comment}{// b = variable of type `int` (built-in type)}
    \textcolor{comment}{//}
    s | b;
  \}
\};

\}\}\} \textcolor{comment}{// end namespace magistrate::intrusive::examples}

\textcolor{keywordtype}{int} main(\textcolor{keywordtype}{int}, \textcolor{keywordtype}{char}**) \{
  \textcolor{keyword}{using namespace }magistrate::intrusive::examples;

  \textcolor{comment}{// Define a variable of custom type `MyTest`}
  MyTest my\_test\_inst(11, 12);
  my\_test\_inst.print();

  \textcolor{comment}{// Call the serialization routine for the variable `my\_test\_inst`}
  \textcolor{comment}{// The output is a unique pointer: `std::unique\_ptr<SerializedInfo>`}
  \textcolor{comment}{// (defined in `src/checkpoint\_api.h`)}
  \textcolor{keyword}{auto} ret = \hyperlink{namespacecheckpoint_a075da4e7344cf037943362517e606c3a}{checkpoint::serialize}(my\_test\_inst);

  \{
    \textcolor{comment}{// Display information about the serialization "message"}
    \textcolor{keyword}{auto} \textcolor{keyword}{const}& buf = ret->getBuffer();
    \textcolor{keyword}{auto} \textcolor{keyword}{const}& buf\_size = ret->getSize();
    printf(\textcolor{stringliteral}{"ptr=%p, size=%ld\(\backslash\)n"}, static\_cast<void*>(buf), buf\_size);
  \}

  \textcolor{comment}{// De-serialization call to create a new unique pointer to `MyTest`}
  \textcolor{keyword}{auto} t = checkpoint::deserialize<MyTest>(ret->getBuffer());

  \textcolor{comment}{// Display the result}
  t->print();

  \textcolor{keywordflow}{return} 0;
\}

\end{DoxyCodeInclude}
