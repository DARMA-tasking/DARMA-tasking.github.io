Using vt in trace-\/only mode\hypertarget{trace-only_what-is-vt-trace}{}\section{What is trace-\/only mode}\label{trace-only_what-is-vt-trace}
Trace-\/only mode is lightweight library which includes the subset of V\+T-\/runtime that allows to generate trace files for an application that uses M\+PI calls. The following examples will show how to generate vt-\/trace target and then how to use it.\hypertarget{trace-only_build}{}\section{Build vt with trace-\/only target\+:}\label{trace-only_build}
Depending on how you build VT, you can either set environment variable {\ttfamily V\+T\+\_\+\+B\+U\+I\+L\+D\+\_\+\+T\+R\+A\+C\+E\+\_\+\+O\+N\+LY}\+: 
\begin{DoxyCode}
export VT\_BUILD\_TRACE\_ONLY=1
/vt/ci/build\_cpp.sh /vt /build
\end{DoxyCode}


or add C\+Make flag {\ttfamily vt\+\_\+trace\+\_\+only}\+:


\begin{DoxyCode}
cmake -Dvt\_trace\_only=1
\end{DoxyCode}


This will generate vt-\/trace target alongside the regular vt-\/runtime C\+Make target.



\hypertarget{trace-only_usage}{}\section{Example usage\+:}\label{trace-only_usage}
\subsection*{C\+Make}

Next step is to setup C\+Make. Find the {\ttfamily vt} project and load its settings\+:


\begin{DoxyCode}
find\_package(vt REQUIRED)
\end{DoxyCode}


Then link {\ttfamily vt-\/trace} library with your target\+:


\begin{DoxyCode}
target\_link\_libraries($\{target\_name\} PUBLIC vt::vt-trace)
\end{DoxyCode}


Example C\+Make\+: 
\begin{DoxyCode}
cmake\_minimum\_required(VERSION 3.10 FATAL\_ERROR)

project(my\_target)

find\_package(vt REQUIRED)

add\_executable(
  my\_target
  $\{CMAKE\_CURRENT\_SOURCE\_DIR\}/src/sample.h
  $\{CMAKE\_CURRENT\_SOURCE\_DIR\}/src/sample.cc
)

target\_link\_libraries(my\_target PUBLIC vt::vt-trace)
\end{DoxyCode}


\subsection*{Source code}

The following code snippet shows the example use of vt-\/trace library (See {\ttfamily \hyperlink{structvt_1_1trace_1_1_trace_lite}{vt\+::trace\+::\+Trace\+Lite}} for more information)\+:


\begin{DoxyCode}
\textcolor{preprocessor}{#include <mpi.h>}
\textcolor{preprocessor}{#include "\hyperlink{trace__lite_8h}{vt/trace/trace\_lite.h}"}

\textcolor{keywordtype}{int} main(\textcolor{keywordtype}{int} argc, \textcolor{keywordtype}{char}** argv) \{
  MPI\_Init(&argc, &argv);

  \textcolor{comment}{// Create vt::trace::TraceLite object after MPI has been initialized}
  \textcolor{comment}{// The constructor takes one argument - your application's name}
  \hyperlink{structvt_1_1trace_1_1_trace_lite}{::vt::trace::TraceLite} myTrace(\textcolor{stringliteral}{"MyApp"});


  \textcolor{comment}{// Call initializeStandalone once to initialize all internal data needed for tracing}
  \textcolor{comment}{// This function takes single argument - MPI communicator}
  myTrace.initializeStandalone(MPI\_COMM\_WORLD);

  \textcolor{comment}{/*}
\textcolor{comment}{     DO MPI RELATED WORK HERE}
\textcolor{comment}{  */}

  \textcolor{comment}{// Use this function to flush traces to file}
  myTrace.flushTracesFile();

  \textcolor{comment}{// After all work is done, call this function to cleanup all data}
  \textcolor{comment}{// This should be called before calling 'MPI\_Finalize'}
  myTrace.finalizeStandalone();

  MPI\_Finalize();
\}
\end{DoxyCode}
 